\documentclass{article}
\usepackage[utf8]{inputenc}
\usepackage{amsfonts, amsmath, amssymb}
\usepackage[english]{babel}
% \usepackage{boisik}
\usepackage{amsthm}
\usepackage[margin=0.5in]{geometry}

%\usepackage{gfsartemisia}
%\usepackage[T1]{fontenc}
\usepackage{mathpazo}
\usepackage[light]{CormorantGaramond}
\usepackage{quiver}
\usepackage{bm}

\usepackage{epigraph}
%\usepackage{tgbonum}
%\usepackage{cmbright}
%\usepackage{textcomp}
\usepackage[object=am]{pgfornament}
\usepackage{graphicx}
\usepackage{tikz-cd}

\usepackage{hyperref} %Uncomment for Hyperlinked Table of Contents.

\hypersetup{
	colorlinks,
	citecolor=blue,
	filecolor=black,
	linkcolor=blue,
	urlcolor=blue
}

\theoremstyle{definition}
\newtheorem{definition}{$\boxed{\star}$ Definition}
\newcommand{\tit}[1]{\textit{#1}}
\newtheorem{theorem}{$\boxed{\boxed{\circledast}}$ Theorem}


\theoremstyle{remark}
\newtheorem*{remark}{Remark}

\theoremstyle{definition}
\newtheorem{corollary}{$ \to $ Corollary}

\theoremstyle{definition}
\newtheorem{proposition}{$\bigstar$ Proposition}

\theoremstyle{definition}
\newtheorem*{attempt}{Attempt}

\title{Category Theory\\%Toposes, Triples \& Theories\\
	\large Definitions, Propositions, Theorems \& Proofs}
\author{Animesh Renanse}
\date{\today}
\usepackage{amsthm}

\newcommand{\inv}[1]{#1^{-1}}
\newcommand{\gen}[1]{\left\langle #1\right\rangle}
\newcommand{\order}[1]{\left\vert #1 \right\vert}
\newcommand{\image}[0]{\text{Im }}
\newcommand{\kernel}[0]{\text{Ker }}
\newcommand{\nsg}[0]{\trianglelefteq}
\newcommand{\isomorph}{\cong}
\newcommand{\End}[1]{\text{\textbf{End}}\left(#1\right)}
\newcommand{\Auto}[1]{\text{\textbf{Aut}}\left(#1\right)}
\newcommand{\pset}{\mathbf{P}}
\newcommand{\proofref}[1]{\emph{Refer to proof in Appendix #1}}
%\makeatletter
%\newcommand*\bigcdot{\mathpalette\bigcdot@{.5}}

%For Categories
\newcommand{\cat}[1]{\mathfrak{#1}}
\newcommand{\opcat}[1]{\mathfrak{#1}^{\text{op}}}
\newcommand{\obj}[1]{\text{\textit{Ob}}(#1)}
\newcommand{\arr}[1]{\text{\textit{Ar}}(#1)}

\newcommand{\Id}[1]{\text{id}_{#1}}
\newcommand{\homset}[3]{\text{Hom}_{#1}(#2,#3)}

\renewcommand{\qedsymbol}{\ensuremath{\blacksquare}}
\newcommand{\point}[0]{$\blacktriangleright\;$}
\newcommand{\singobj}[1]{\bullet_{#1}}
\newcommand{\Func}[2]{\text{Func}\left (#1,#2\right )}
\newcommand{\Nat}[2]{\text{Nat}\left (#1,#2\right )}
\newcommand{\func}[2]{{#2}^{#1}}
\newcommand{\GL}[1]{\text{GL}_{#1}}
\newcommand{\Un}{\text{Un}}
\newcommand{\elem}[1]{\in ^{#1}}

\newcommand{\res}{$ \bigstar $ \textbf{Proposition.}\;}
\newcommand{\Cone}[2]{\text{Cone}\left (#1,#2\right )}
\newcommand{\Cocone}[2]{\text{Cocone}\left (#1,#2\right )}
\newcommand{\ulg}[1]{\left \vert #1\right\vert }
\newcommand{\Eq}[2]{\text{Eq}\left (#1,#2\right )}
\newcommand{\Coeq}[2]{\text{Coeq}\left (#1,#2\right )}
\newcommand{\src}[1]{\;\text{src}\left (#1\right )}
\newcommand{\tar}[1]{\;\text{tar}\left (#1\right )}
\newcommand{\colim}{\text{colim}}
\newcommand{\ang}[1]{\left \langle #1 \right \rangle}

\begin{document}
	\maketitle
	\tableofcontents
	\newpage
\section{Categories}
A category $ \cat{C} $ is a collection of two entities : \textbf{Objects}, denoted by $ \obj{\cat{C}} $ and \textbf{Arrows}, denoted by $ \arr{\cat{C}} $. Each arrow in $ \arr{\cat{C}} $ is assigned a \textbf{Domain} and a \textbf{Co-Domain} from the $ \obj{\cat{C}} $ denoted by
\[f \in \arr{\cat{C}} \;\text{such that}\;f : A \to B\;\text{for}\;A,B\in \obj{\cat{C}}.\]
where $ A $ is the domain of arrow $ f $ and $ B $ is the co-domain. Moreover, if it so happens that there are arrows $ f,g\in \arr{\cat{C}} $ such that $ f : A \to B$ and $ g : B\to C $, then there must be another arrow, called \textbf{Composite} $ g\circ f \in \arr{\cat{C}}$ such that $ g\circ f : A\to C $. Finally, for each object $A\in \obj{C} $, there must be an \textbf{Identity} arrow $ \arr{\cat{C}} $ such that $ \Id{A} : A\to A $.\\\\
All the arrows must confirm to \textbf{Associativity under composition} and \textbf{Identity over composition with Identity arrows}:
\begin{itemize}
	\item{For $ f : A\to B,g : B\to C, h : C\to D $ in $ \arr{\cat{C}} $ for $ A,B,C,D \in \obj{\cat{C}} $,
\[h\circ (g\circ f) = (h\circ g)\circ f\]	
}
\item{For any $ f : A\to B $ 
\[f\circ \Id{A} = \Id{B}\circ f = f\]
}
\end{itemize}
\point\textbf{Small Categories} : When $ \arr{\cat{C}} $ is a set. ($ \implies $ The $ \obj{\cat{C}} $ is also a set.)\\\\
\point\textbf{Homset} : Select two $ A,B\in \obj{\cat{C}} $ and then the (\textit{assumed}) set of all arrows $ f : A\to B $ in $ \arr{\cat{C}} $ is represented by $ \homset{\cat{C}}{A}{B} $.\\\\
\point\textbf{Subcategory} : A subcategory $ \cat{D} $ of a category $ \cat{C} $ is a pair of subsets $ D_O $ and $ D_A $ of the objects and arrows of $ \cat{C} $ respectively with the property that (1.) any $ f\in D_A $ must have it's source and target in $ D_O $, (2.) if $ C\in D_O $ then $ \Id{C}\in D_A $ and (3.) if $ f,g \in D_A$ are composable then their composite $ g\circ f $ must also be in $D_A$.\\\\
\point \textbf{Product Category} : Consider two categories $ \cat{C} $ and $ \cat{D} $. One can construct another category $ \cat{C}  \times \cat{D}$ called product category for which:
\begin{itemize}
	\item{\textbf{Objects} are the pairs $ (A,B) $ where $ A \in \obj{\cat{C}} $ and $ B \in \obj{\cat{D}} $.}
	\item {\textbf{Arrows} are the pairs $ (f,g) $ where $ f : A\to A^{\prime} $ and $ g : B\to B^{\prime} $ so that we write
\[(f,g) : (A,B) \longrightarrow (A^{\prime}, B^{\prime})\]
as an arrow of $ \cat{C}\times \cat{D} $.	
} 
\item {\textbf{Composition} is defined component-wise. That is, for two composable arrows $ (f,g) $ and $ (h,k) $, we can write their composite as
\[(f,g)\circ (h,k) = (f\circ h, g\circ k).\]
}
\end{itemize}
\point\textbf{Slice Category} : A category $ \cat{C}/A $ of objects of $ \cat{C} $ \textit{over} an object $ A $ such that $ \obj{\cat{C}/A} $ is the collection of all arrows of $ \cat{C} $ with target $ A $ and for objects $ f:B\to A $ and $ g:C\to A $, there is an arrow $ h :B\to C$ in $ \arr{\cat{C}/A} $ such that $ f = g\cdot h $.
\begin{figure}[h!]
	\[\begin{tikzcd}
		B\arrow[swap]{d}{f}\arrow{r}{h}  &C \arrow{dl}{g}\\
		A
	\end{tikzcd}\]
\caption{$ h : B\to C $ is the arrow in $ \arr{\cat{C}/A} $ with source $ f $ and target $ g $.}
\end{figure}
\newline
\point \textbf{Isomorphism} : Arrow $ f : A\to B $ in a category is an isomorphism if it has an \emph{inverse} arrow $ g:B\to A $ such that $ f\circ g = \Id{B} $ and $ g\circ f: \Id{A} $. Moreover, $ A $ and $ B $ are then referred isomorphic.\\\\
\point \textbf{Opposite Category} : Category $ \opcat{C} $ is the opposite category of $ \cat{C} $ if for any $ A,B\in \obj{\cat{C}}$ we have
\[\homset{\cat{C}}{A}{B} = \homset{\opcat{C}}{B}{A}\]
\point \textbf{Initial \& Terminal Object} : $ T\in \obj{\cat{C}} $ is terminal (initial) if $ \homset{\cat{C}}{A}{T} $ ($ \homset{\cat{C}}{T}{A} $) has one element only, for any $ A\in \obj{\cat{C}} $. 

\hrulefill
\subsection{Selected problems in Exercise 1.1}  
\begin{attempt}[SGRPOID]
	\textit{Given} : Define $ e $ to have the \emph{identity property} when for all $ f$ and $ g $, $ e\circ f =  f $ whenever $ e\circ f $ is defined and $ g\circ e = g $ whenever $ g\circ e $ is defined. \\
	Consider that we have a category $ \cat{C} $. We now show that the given alternate definition is equivalent to the usual definition. First consider the information we get from part a. The first equivalence yields that composition is defined between the \textit{composable} arrows from collection of arrows $ \arr{\cat{C}} $. Second \& Third implies the existence of dual composes which hence implies that target of $ h$ = source of $ g $ and target of $ g $ = source of $ f $. This sets the setting for associativity rule.\\
	Next, the part b implies the \textit{associativity} rule of arrows composable in $ \arr{\cat{C}} $.\\
	Part c implies the \textit{existence of identity} elements $ e $ and $ e^\prime $ which corresponds to the identity arrows of target and source of $ $ respectively.
\end{attempt}
\begin{attempt}[CCON]
	\label{CCON}
	\textit{Given} : The proposed categorical constructions.\\
Let $ \cat{C} $ be a category. We now verify the corresponding properties indeed lead to categories.\\
The part a introduces \textit{arrow category} $ \cat{Ar}(\cat{C})$ which has objects as arrows of $ \cat{C} $ and an arrow from $ f :A\to B $ to $ g:A^\prime \to B^\prime $ becomes a pair of arrows $ h:A\to A^\prime $ and $ k : B\to B^\prime $ such that $ g\circ h  = k\circ f$. Clearly, each arrow in $ \arr{\cat{Ar}(\cat{C})} $ has a source and target as arrows in $ \obj{\cat{Ar}(\cat{C})} $. Consider four arrows $ k:A\to B , g:B\to C, f:C\to D\;\&\; d:E\to F$ in $ \obj{\cat{Ar}(\cat{C})} $. Note that these are arrows of $ \cat{C} $. We then have that there are 6 arrows $ h_1:A\to B, k_1 : B\to C, h_2: B\to C, k_2:C\to D, h_3 : C\to E\;\& \; k_3 : D\to F $ in $ \arr{\cat{Ar}(\cat{C})} $. Since the arrows $ h_1,k_1,h_2,k_2,h_3,k_3 $ are associative as they are arrows of $ \cat{C} $, hence we have the \textit{associativity} for $ \cat{Ar}(\cat{C}) $.\\
The \textit{identity} and \textit{unital} property is trivial from the identity arrows of category $ \cat{C} $ and the given commutative diagram.
\end{attempt}
\hrulefill
\newpage
\section{Functors}
If $ \cat{C} $ and $ \cat{D} $ are categories, then a functor $ F : \cat{C} \to \cat{D} $ is a map between categories for which,
\begin{itemize}
	\item{Functors maps \textit{arrows to arrows} : If $ f:A\to B $ is an arrow in $ \arr{\cat{C}} $, then $ Ff : FA \to FB $ is an arrow in $ \arr{\cat{D}} $.}
	\item {Functors \textit{preserves identity arrows} : $ F(\Id{A}) =  \Id{FA}$.}
	\item {Functors \textit{preserves composition} : If $f: A\to B $ and $ g: B\to C $ are in $ \arr{C} $ then $ F(g\circ f) = Fg \circ Ff $.}
\end{itemize}
\point \textbf{Contravariant Functor} : A functor $ F:\opcat{C}\to \cat{D} $ is called contravariant from $ \cat{C} $ to $ \cat{D} $. The usual functor $ F:\cat{C} \to \cat{D}$ is called covariant.\\\\
\point\textbf{Faithful Functor} : A functor $ F:\cat{C}\to \cat{D} $ is faithful if it is injective when restricted to each homset. That is, each arrow in any homset is mapped to a distinct arrow in the mapped homset.\\\\
\point \textbf{Full Functor} : A functor $ F:\cat{C}\to \cat{D} $ is full if it is surjective on each homset. That is, for any $ A,B\in \obj{\cat{C}} $, every arrow in $ \homset{\cat{D}}{FA}{FB} $ is mapped by $ F $ from some arrow in $ \homset{\cat{C}}{A}{B} $.\\\\
\point \textbf{Preservation of a property} : A functor $ F $ preserves a property $ P $ that an arrow may have if $ F(f) $ also has the property $ P $ whenever $ f $ has.\\\\
\point \textbf{Reflection of a property} : A functor $ F $ reflects a property $ P $ if $ f $ has the property whenever $ F(f) $ has.\\\\
\point \textbf{Hom Functor} : First, for a fixed object $ A $, define $ \homset{\cat{C}}{A}{f}  : \homset{\cat{C}}{A}{B} \longrightarrow \homset{\cat{C}}{A}{C}$ for each $ f:B\longrightarrow C $ by requiring that $ \homset{\cat{C}}{A}{f}(g) = f\circ g$ for any $ g\in \homset{\cat{C}}{A}{B}$. Now, 
\begin{enumerate}
	\item {For a fixed object $ A \in \obj{\cat{C}}$, $ \homset{\cat{C}}{A}{-} : \cat{C} \longrightarrow \cat{Set}$ is a \textbf{covariant hom functor}.}
	\item {For a fixed object $ B\in\obj{\cat{C}}  $, $ \homset{\cat{C}}{-}{B} : \opcat{C}\longrightarrow \cat{Set}$ is a \textbf{contravariant hom functor}.}
\end{enumerate}
\emph{Refer to proof in Appendix \ref{PROOF-1}}.\\\\
\point \textbf{Powerset Functor} : The powerset contravariant functor is defined as the $ \pset : \opcat{Set} \longrightarrow \cat{Set} $ such that if $ f : A\to B\in \arr{\cat{Set}} $, then $ \pset f : \pset B \to \pset A $ is the inverse of $ f $. Specifically, for $ B_0 \in \pset B $ (that is, $ B_0 \subseteq B $), we have
\[\pset f(B_0) = \{x\in A \;\vert\; f(x) \in B_0\} = \inv{f}(B_0).\]
\subsection{Equivalence of Categories}
Since the composite of a functor is a functor, the collection of categories is in itself a category, denoted by $ \cat{Cat} $(!) \proofref{\ref{PROOF-2}}.\\\\
\point \textbf{Isomorphic Categories} : If $ \cat{C} $ and $ \cat{D} $ are categories and $ F $ is a functor $ F : \cat{C} \to \cat{D} $ such that it is an isomorphism in category of categories $ \cat{Cat} $, then $ \cat{C}$ and $ \cat{D} $ are said to be isomorphic.\\\\
\point \textbf{Functor Equivalence} : A functor $ F : \cat{C} \longrightarrow \cat{D} $ is said to be an equivalence if it's full and faithful and has the property that for any object $ B \in \obj{\cat{D}}$, there is an object $ A\in \obj{\cat{C}} $ for which $ F(A) $ is isomorphic to $ B $. Moreover, $ \cat{C} $ and $ \cat{D} $ are said to be equivalent.\\\\
\point \textbf{Comma Categories} : Let $ \cat{A},\cat{C} $ and $ \cat{D} $ be categories and $ F : \cat{C} \to \cat{A} $, $ G:\cat{D} \to \cat{A}$ be functors. The comma category $ (F,G) $ is the generalization of \emph{slice categories} $ \cat{A}/ A $ over an object $ A $. 
\begin{itemize}
	\item {The \textbf{objects} of $ (F,G) $ are triples $ (C,f,D) $ with $ f: FC \to GD$ an arrow in $ \arr{\cat{A}} $, $ C \in \obj{\cat{C}} $  and $ D\in \obj{\cat{D}} $.}
	\item {An \textbf{arrow} $ (h,k): (C,f,D) \longrightarrow (C^{\prime},f^{\prime},D^{\prime}) $ consists of $ h:C\to C^{\prime} $ and $ k:D\to D^{\prime} $ with the property that 
\[\begin{tikzcd}
	FC\arrow{r}{Fh} \arrow[swap]{d}{f} &FC^{\prime} \arrow{d}{f^{\prime}}\\
	GD\arrow[swap]{r}{Gk} &GD^{\prime}
\end{tikzcd} \;\;\;\text{\textbf{commutes}.}\]	
}
\item {\textbf{Composition} of arrows is given on components by composition in $ \cat{C} $ and $ \cat{D} $. That is, when conformable, $ (h_1,k_1)\circ (h_2,k_2) = (h_1\circ h_2,k_1\circ k_2) $.}
\end{itemize}
From the definition, it's easy to see $ (F,G) $ forms a category. Moreover, if $ \cat{C} = \cat{A} $ and $ \cat{D} = \cat{1} $, then $ F $ would be an identity functor and $ G $ would map $ \singobj{\cat{1}} $ to some object $ A_\bullet $ in $ \cat{A} $. Then $ (\Id{\cat{A}},A_\bullet) $ is exactly the slice category $ \cat{A}/A_\bullet $, which is easy to see from the above diagram too.\\\\
\point \textbf{Projections} : Each comma category $ (F,G) $ is equipped with two projections:
\begin{enumerate}
	\item {$ p_1 : (F,G) \to \cat{C} $ which projects objects and arrows of $ (F,G) $ onto their first coordinates. That is, $ (C,f,D) $ to $ C $ and $ (h,k) $ to $ h $.}
	\item {$ p_2 : (F,G)  \to \cat{D}$ which projects objects onto their third coordinates and arrows onto their second. That is, $ (C,f,D) $ to $ D $ and $ (h,k) $ to $ k $.}
\end{enumerate}

\hrulefill 
\subsection{Selected problems in Exercise 1.2}
\begin{attempt}[PISO]
	Consider categories $ \cat{C} $ and $ \cat{D} $ and let $ F $ be a Functor between them. Then if $ f\in \arr{\cat{C}} $ is an isomorphism such that $ f: A\to B $, then there exists $ g : B\to A $ such that $ f\circ g = \Id{B} $ and $ g\circ f = \Id{A} $. Now, by the definition of functor, $ Ff : FA\to FB $ and $ Fg : FB\to FA $ with $ F(f\circ g) = Ff\circ Fg = F\Id{B} = \Id{FB} $ and $ F(g\circ f) = Fg\circ Ff = F\Id{A} = \Id{FA} $. Therefore $ Ff $ is also an isomorphism in category $ \cat{D} $.\\
	Now, consider $ \cat{D} $ be such that it contains only one object and one arrow. Now a functor from any category $ \cat{C} $ with atleast one arrow which is not an isomorphism, will map that arrow to an isomorphism in $ \cat{D} $, showing that functors may not reflect the isomorphism property.
\end{attempt}
\begin{attempt}[EAAM] \emph{In Notes. Add Later.}
	
\end{attempt}
\begin{attempt}[PREORD]
	\emph{In Notes. Add Later.}
\end{attempt}
\hrulefill
\newpage
\section{Natural Transformations}
\point \textbf{Natural Transformation} : If $ F : \cat{C} \to \cat{D} $ and $ G : \cat{C} \to \cat{D} $ are two functors, then $ \lambda : F \longrightarrow G $ is a natural transformation from $ F $ to $ G $ if $ \lambda $ is a collection of arrows $ \lambda C : FC \longrightarrow  GC $, one for each object $ C \in \obj{\cat{C}} $, such that for each arrow $ g : C\longrightarrow C^{\prime} $ in $ \arr{\cat{C}} $ the following diagram
\[\begin{tikzcd}
	FC \arrow[swap]{d}{Fg} \arrow{r}{\lambda C} &GC\arrow{d}{Gg}\\
	FC^{\prime} \arrow[swap]{r}{\lambda C^{\prime}} &GC^{\prime} 
\end{tikzcd}\;\;\text{\textbf{commutes}.}\]
Also, the arrows $ \lambda C $ are known as the \textbf{components} of $ \lambda $.\\
An example is the determinant map from $ \text{GL}_n $ to $ \text{Un} $. \emph{Refer to concrete statement and proof in Appendix \ref{PROOF-3}}.\\\\
\point \textbf{Natural Equivalence} : The natural transformation $ \lambda $ is a natural equivalence if each component of $ \lambda $ is an isomorphism.\\\\
\point \textbf{Functor Category} : Let $ \cat{C} $ and $ \cat{D} $ be categories where $ \cat{C} $ is small. The collection $ \Func{\cat{C}}{\cat{D}} $ of functors from $ \cat{C} $ to $ \cat{D} $ as objects with natural transformations between them as arrows where composition $ \mu\circ \lambda $ between two arrows $ \mu: F\to G $ and $ \lambda : G\to H $ (natural transformations) is defined by requiring it's component at $ C $ to be $ \mu C \circ \lambda C $, is a category, named as Functor Category.\\ 
Few basic observations are:
\begin{itemize}
	\item {The functor category $ \obj{\Func{\cat{C}}{\cat{D}} }$ is just the $ \homset{\cat{Cat}}{\cat{C}}{\cat{D}} $.}
	\item {Due to the above observation, we can extend the idea of Hom functors to $ \Func{\cat{C}}{\cat{D}} $. That is, for any $ F : \cat{D} \to \cat{E} $, 
\[\Func{\cat{C}}{F} : \Func{\cat{C}}{\cat{D}} \longrightarrow \Func{\cat{C}}{\cat{E}}\]
is a Functor, not only a morphism in $ \cat{Set} $.	
}
\item {The Hom-Functor in $ \Func{\cat{C}}{\cat{D}} $ is denoted by $ \Nat{F}{G} $ for functors $ F,G : \cat{C} \to \cat{D}$ in $ \obj{\Func{\cat{C}}{\cat{D}}} $.}
\item {Alternatively, one can write $ \Func{\cat{C}}{\cat{D}} $ as $ \func{\cat{C}}{\cat{D}} $.}
\end{itemize}
\emph{A nice exercise is presented in Appendix \ref{A-6} (\textbf{Complete it Please}).}
\subsection{Selected problems from Exercise 1.3}
\begin{attempt}{(NTF).}  
We need to describe a natural transformation as a functor from an arrow category to a functor category. Define the following map:
\[\Omega :\cat{Ar(C)} \longrightarrow \Func{\cat{I}}{\cat{C}}\]
where $ \obj{\cat{I}} = \{A,B\}  $ and $ \arr{\cat{I}} =\{\Id{A}, \Id{B}, f : A\to B\} $. The map $ \Omega  $ is such that for $ g : C\to D$ in $ \obj{\cat{Ar(C)}} $, we get $ \Omega g = F_g $ where $ F_g $ is a functor from $ \cat{I} $ to $ \cat{C} $. Due to the structure of $ \cat{I} $, we get that $ F_g f : F_g A \to F_g B $ is a unique arrow of $ \Func{\cat{I}}{\cat{C}} $ and each arrow of $ \cat{C} $ can hence be mapped to a unique arrow in $ \Func{\cat{I}}{\cat{C}} $. Now, $ \Omega $ is also defined to take an arrow $ (a,b) $ in $ \cat{Ar(C)} $ to component-wise action, that is, $ \Omega (a,b) = (\Omega a, \Omega b) $, where $ (a,b) $ is such that
\[\begin{tikzcd}
	C \arrow[swap]{d}{a}\arrow{r}{g} &D\arrow{d}{b}\\
	E\arrow[swap]{r}{h} &F
\end{tikzcd}\;\text{commutes.}\] 
Now, for two functors $ F_g $ and $ F_h $ in $ \Func{\cat{I}}{\cat{C}} $, generated from $ \Omega g $ and $ \Omega h $ for $ g,h \in \obj{\cat{Ar(C)}} $, we get 
\[\begin{tikzcd}
	F_g A \arrow[swap]{d}{F_a f}\arrow{r}{F_gf} &F_g B\arrow{d}{F_bf}\\
	F_h A \arrow[swap]{r}{F_hf} &F_h B
\end{tikzcd}\]
which also commutes because $ \Omega $ with the above definition forms an equivalence functor. Hence, there is an equivalence between $ \cat{Ar(C)} $ and $ \Func{\cat{I}}{\cat{C}} $. \footnote{Note that to make the above construction simpler, once could have simply defined $ \Omega g = F_g  $ as $ F_g(A) = C, F_g(B) = D $ and $ F_gf = g $ for any $ g:C\to D $ in $ \obj{\cat{Arr(C)}} $. It is easily verified that this construction makes $ \Omega g $ a functor from $ \cat{I}\to \cat{C} $ and hence, $ \Omega $ itself being a functor.}
\end{attempt}
\begin{attempt}{(NTG).} 
	It's not difficult to see that the natural transformation between groups as one element categories are the collection of elements of the target group which commute with all other elements of the group (or inner automorphism of the group) due to naturality conditions of the commutative square.
\end{attempt}
\newpage
\section{Elements}
\point \textbf{Element of $ A $ defined over $ T $} : Consider a category $ \cat{C} $ and an arrow $ x : T \to A $ in $ \arr{\cat{C}} $. Then, the arrow $ x $ can be considered as element of $ A $, defined over $ T $\footnote{One way of thinking of an element $ x $ is to consider $ x $ as a set of elements of object $ A $ indexed by object $ T $ in a category.}. 
\begin{itemize}
	\item {When $ x:T\to A $ is thought of as an element of $ A $ defined over $ T $, then we call $ T $ as the \textbf{domain of variation} of element $ x $.}
	\item {One denotes element $ x:T\to A $ of $ A $ defined over $ T $ as $ x\elem{T}A $.}
	\item {If $ x\elem{T} A $ and $ f : A\to B $ is an arrow, then $ f\circ x \elem{T} B $. It's hence simpler to write $ f\circ x $ as $ f(x) $.}
	\item {The element $ \Id{A} : A\to A $ is called the \textbf{generic element} of $ A $.}
	\item {If $ F : \cat{C} \to \cat{D}$ is a functor and $ A $ is an object of $ \cat{C} $, then \textbf{functor} $ F $ maps $ A $ to an object $ FA $ in $ \cat{D} $ such that 
\begin{enumerate}
	\item {If $ x = \Id{A} : A\to A $ is generic, then $ Fx = \Id{FA} : FA \to FA$ is also generic.}
	\item {If $ x\elem{T}A $ and $ f : A\to B $, then $ F(f(x)) = Ff\circ Fx$.}
\end{enumerate}	
Both are just the usual properties of a functor.
}
\item {An arrow $ f : A\to B $ is an \textbf{isomorphism} in a category $ \cat{C} $ if and only if $ f $, thought of as a function, is a bijection between the elements of $ A $ defined over $ T $ and elements of $ B $ defined over $ T $ for all objects $ T $ of $ \cat{C} $.}
\item {A \textbf{terminal object} $ A $ can hence be seen to contain only one element for any domain of variation $ S $. That is, if $ A $ is terminal, then $ x\elem{S}A $ is the only element of $ A $ with this domain of variation.}
\end{itemize}
\subsection{Monic \& Epic} 
\point \textbf{Monomorphism} : An arrow $ f :A\to B $ in a category $ \cat{C} $ is called a monomorphism if $ f $ is injective on elements of $ A $ defined over any object $ T $. That is, for $ x,y\elem{T} A $, $ f(x) = f(y) $ implies $ x=y $. Note that we actually here mean $ \homset{\cat{C}}{T}{f} $ when we write$ f $.\\
Another important observation is that since $ f $ is injective, $ f\circ x = f\circ y \implies x = y $ for $ x,y \elem{T} A $, which is that a \emph{mono is left cancellable.} Denote $ f $ to be a monomorphism by:
\[\begin{tikzcd}
	A & B
	\arrow[tail, from=1-1, to=1-2]
\end{tikzcd}\]
\\
\point  \textbf{Epimorphism} : The arrow $ f : A\to B$ is an epimorphism if it is \emph{right cancellable}. That is, if $ x\circ f = y\circ f $, then $ x=y $ for $ x,y \elem{B} C $. This is true if and only if the contravariant hom functor $ \homset{\cat{C}}{f}{T} $ is injective for any object $ T $. With this, it is trivial to see that a monic is injective and epic is surjective in $ \cat{Set} $\footnote{The surjective nature of epis follows from the inverse map $ \homset{\cat{C}}{f}{T} : \homset{\opcat{C}}{B}{T} \to \homset{\opcat{C}}{A}{T} $ which is given injective. Remember contravariant hom functor is given by right composition in $ \opcat{C} $, that is, $ \homset{\cat{C}}{f}{T} : \opcat{C} \to \cat{Set} $ such that $ \homset{\cat{C}}{f}{T} (g) = g\circ f$ for $ g\in \homset{\opcat{C}}{B}{T} $. }. Denote $ f  $ to be a epimorphism by:
\[\begin{tikzcd}
	A & B
	\arrow[two heads, from=1-1, to=1-2]
\end{tikzcd}\]
\\
\point \textbf{Split Epimorphism} : An epimorphism $ f : A\to B $ in $ \cat{C} $ is a split epimorphism if it is surjective on elements defined over any object $ T $. That is, the set function $ \homset{\cat{C}}{T}{f} : \homset{\cat{C}}{T}{A} \to \homset{\cat{C}}{T}{B} $ defined by $ \homset{\cat{C}}{T}{f} (g) = f\circ g$ for $ g\in 
\homset{\cat{C}}{T}{A} $ is surjective for any object $ T $ in $ \cat{C} $.\\
Equivalently, if there is an arrow $ g : B\to A $ such that $ f\circ g = \Id{B} $, then $ f $ is a split epimorphism. With this definition, it is trivial to see that all split epis are surjective in $ \cat{Set} $.
\\\\
\point \textbf{Split Monomorphism} : An arrow $ f : A\to B $ with a left inverse $ g : B\to A $ is called a split monos. That is, $ g\circ f = \Id{A} $. Split monos in $ \cat{Top} $ are called \emph{retractions}.
\subsection{Subobject}
\point \textbf{Factors of an arrow} : Suppose $ a : T \to A $ and $ i : A_0 \to A $ is a monomorphism, then, if there exists an arrow $ j: T\to A_0 $ such that
\[\begin{tikzcd}
	T & A \\
	{A_0}
	\arrow["a", from=1-1, to=1-2]
	\arrow["j"', from=1-1, to=2-1]
	\arrow["i"', tail, from=2-1, to=1-2]
\end{tikzcd}\;\text{\textbf{commutes},}\]
then we say that \emph{$ a $ factors through $ i $}. We additionally write that $ a\elem{T}_A A_0 $ to say that \emph{the element $ a  $ of $ A $ is an element of $ A_0 $.}
\newpage
\hrulefill
\begin{proposition}\label{P-1}
	Let $i :A_0 \to A $ and $ i^{\prime} : A_0^{\prime} \to A $ be monomorphisms in a category $ \cat{C} $. Then $ A_0 $ and $ A_0^{\prime} $ have the same elements of $ A $ if and only if there is an \textbf{isomorphism} $ j : A_0 \to A_0^{\prime} $ such that
	\[\begin{tikzcd}
		{A_0^{\prime}} & A \\
		{A_0}
		\arrow[tail,"{i}"', from=2-1, to=1-2]
		\arrow[tail,"i^{\prime}", from=1-1, to=1-2]
		\arrow["j", from=2-1, to=1-1]
	\end{tikzcd}\;\text{\textbf{commutes.}}\]
\end{proposition}
\begin{proof}
	First, suppose that given to us are the two monomorphisms $ i : A_0 \to A, i^{\prime} : A_0^{\prime}\to A $ and that $ A_0 $ and $ A $ have same elements under $ A $. Since we have $ i \elem{A_0}_A A_0 $ and $ A_0 $ and $ A_0^{\prime} $ have same elements, therefore $ i\elem{A_0}_A A_0^{\prime} $. This means that there is an arrow $ j : A_0 \to A_0^{\prime} $. But we have a monomorphism $ i^{\prime} : A_o^{\prime} \to A $, therefore, $ i $ factors through $ i^{\prime} $. This means that $ i = i^{\prime} \circ j $. Similarly, when we use this argument for $ A_0^{\prime} $ instead of $ A_0 $, we would get another arrow $ k : A_0^{\prime} \to A_0 $ such that $ i^{\prime} = i\circ k $. Combining both $ i = i^{\prime}\circ j $ and $ i^{\prime} = i\circ k $ with the fact that $ i $ and $ i^{\prime} $ are monomorphisms, we get that $ j\circ k = \Id{A_0^{\prime}} $ and $ k\circ j = \Id{A_0}$, showing that $ j $(or $ k $) is an isomorphism.\\
	For the converse, we have that there is an isomorphism $ j : A_0 \to A_0^{\prime} $ such that $ i = i^{\prime}\circ j  $ where $ i : A_0 \to A $ and $ i^{\prime} : A_0^{\prime} \to A $ are given monomorphisms. We need to prove that $ A_0 $ and $ A_0^{\prime} $ have same elements of $ A $. For this, consider the element $ a $ of $ A $ which is also an element of $ A_0^{\prime} $, that is, $ a\elem{T}_A A_0 $. This means that $ a = i\circ u $ where $ u: T\to A_0 $. But we have the isomorphism $ j : A_0 \to A_0^{\prime} $, therefore we can write $ a \elem{T}_A A_0^{\prime} $ where $ a = i^{\prime}\circ j\circ u $. Therefore $ a $ factors through $ A_0^{\prime} $. One can similarly show that for $ a\elem{T}_A A_0^{\prime} $, $ a\elem{T}_A A_0 $, so that both $ A $ and $ A_0^{\prime} $ has same elements of $ A $.
\end{proof}
\hrulefill

\point \textbf{Equivalent Monomorphisms} : Two monomorphisms are said to be equivalent if they have same elements\footnote{By Proposition \ref{P-1}, one can equivalently say that $ a : A\to B $ and $ c : C\to B $ are \emph{equivalent} if and only if $ \exists $ an isomorphism $ j : A \to C $.}.\\\\
\point \textbf{Subobject} : A subobject of an object $ A $ is an equivalence class of monomorphisms into $ A $.\\
 That is, for $ f : A_0 \to A $ is an element of a subobject, then for any other member of that subobject $ g : A_1\to A $, there exists an isomorphism $ j : A_0 \to A_1 $\footnote{It is usual to refer to a subobject of an object $ A $ by referring to one of it's members, say, $ f : A_0 \rightarrowtail A_1 $}. \\
 \emph{More clear statement in Appendix \ref{A-4}}.\\\\
\point \textbf{Global Element} : An arrow in a category $ \cat{C} $ from \textbf{the} terminal object\footnote{Let $ \cat{C} $ be a category. Observe that any arrow which has source as a terminal object is a monomorphism. Moreover, because terminal objects as a target has only one arrow for any source, therefore any two terminal objects in a category are isomorphic! Hence, any arrow from a terminal object is a monomorphism and therefore defines a subobject of it's target which contains only one monomorphism based on each terminal object to that target.} to some object $ A $ is called a global element of $ A $.\\\\

\newpage
\section{Yoneda Lemma}
\point \textbf{Yoneda Embedding} : Consider a category $ \cat{C} $. For an arrow $ f : A\to B $ in $ \cat{C} $, we have two corresponding hom-functors $ \homset{}{B}{-} $ and $ \homset{}{A}{-} $ from $ \cat{C}  $ to $ \cat{Set} $.\\
 The Yoneda embedding for the arrow $ f $ is then defined as the natural transformation $ \lambda : \homset{}{B}{-} \longrightarrow \homset{}{A}{-}$ whose component at an object $ C $ of $ \cat{C} $ is the arrow $ \lambda C : \homset{}{B}{C} \to \homset{}{A}{C} $ which takes an arrow $ g : B\to C $ to $ g\circ f : A\to C$. \\
 Yoneda Embedding therefore defines a contravariant functor $ \lambda : \opcat{C} \to \Func{\cat{C}}{\cat{Set}}$.\\
 \emph{Refer to proof that this is indeed a Functor in Appendix \ref{A-5}}\\
 \begin{proposition}\label{P-2}
 	(\textbf{Yoneda Lemma}) Consider the following two functors:
 	\begin{enumerate}
 		\item {Composite Functor : The following map
 			\[\gamma : \cat{C} \times \Func{\cat{C}}{\cat{Set}} \longrightarrow \cat{Set}\]
 			is a functor which maps 
 			\begin{itemize}
 				\item {\textbf{Objects} : for $ B \in \obj{\cat{C}} $ and functor $ F \in \obj{\Func{\cat{C}}{\cat{Set}}} $,$\gamma$ maps $ (B,F) $ to the following set:
 			\[\gamma (B,F) = \Nat{\homset{}{B}{-}}{F}.\]	
 			}
 		\item {\textbf{Arrows} : for $ f : A\to B $ in $ \arr{\cat{C}} $ and natural transformation $ \lambda : F \to G $ in $ \arr{\Func{\cat{C}}{\cat{Set}}}$, $ \gamma $ maps $ (f,\lambda) $ to the following set function\footnote{Note that this composes the Yoneda Embedding Functor from $ \cat{C} \to \Func{\opcat{C}}{\cat{Set}}$.}:
 	\[\gamma(f,\lambda) (\mu) = \lambda \circ \mu \circ \homset{}{f}{-} \]
 	where $ \mu \in \Nat{\homset{}{A}{-}}{F}$.	Note that $ \homset{}{f}{-} $ is a natural transformation $ \homset{}{B}{-} \longrightarrow  \homset{}{A}{-}$, which is the Yoneda Embedding.
 	}
 			\end{itemize}
 		\item {Evaluation Functor : The following map
 	\[\chi: \cat{C} \times \Func{\cat{C}}{\cat{Set}} \longrightarrow \cat{Set}\]	
 	is a functor which maps
 	\begin{itemize}
 		\item {\textbf{Objects} : for $ B \in \obj{\cat{C}} $ and functor $ F \in \obj{\Func{\cat{C}}{\cat{Set}}} $,$\gamma$ maps $ (B,F) $ to the following set:
 	\[\chi(B,F) = FB.\]	
 	}
	\item {\textbf{Arrows} : for $ f : A\to B $ in $ \arr{\cat{C}} $ and natural transformation $ \lambda : F \to G $ in $ \arr{\Func{\cat{C}}{\cat{Set}}}$, $ \gamma $ maps $ (f,\lambda) $ to the following set function:
\[\chi(f,\lambda) = Gf\circ \lambda A.\]	
 }
 	\end{itemize}
 	}
 	}
 	\end{enumerate}
 Then, for any $ B\in \obj{\cat{C}} $, the arrow in $ \cat{Set} $
 \begin{equation}\label{Eq-1}
 	\begin{split}
 		\varphi :\; &\Nat{\homset{}{B}{-}}{F} \longrightarrow FB\\
 		&\text{which takes a } \lambda \text{ to a set element of $ FB $ as},\\
 		 & \varphi(\lambda) = (\lambda B) (\Id{B})
 	\end{split}
 \end{equation}
is an \textbf{Isomorphism}(!)\\
More specifically, the natural transformation $ \phi $ given as
\begin{equation}
	\phi : \gamma \longrightarrow \chi
\end{equation}
which takes any object $ (B,F) $ in $ \cat{C}\times \Func{\cat{C}}{\cat{Set}} $ to an arrow $ \phi B = \varphi : \Nat{\homset{}{B}{-}}{F} \to FB$ as described above, is a \textbf{Natural Equivalence}.
 \end{proposition}
\newpage
\begin{proof}
	\textbf{Act 1.} : Let us first show that $ \varphi $ in \eqref{Eq-1} is indeed an isomorphism. \\\\
	Define the following map 
	\[I : FB \longrightarrow \Nat{\homset{}{B}{-}}{F}\]
	which maps each set element $ k\in FB $ to the natural transformation $ \mu $ such that for any object $ A $ of category $ \cat{C} $ and for any subsequent arrow $ g \in \homset{}{B}{A} $, $ \mu $ must satisfy
	\[(\mu A)g = Fg(k).\]
	Note that such a map always defines a natural transformation because, for a natural transformation $ \lambda $ between $ \homset{}{B}{-} $ and $ F $, we must have that
\[\begin{tikzcd}
	FC & FA \\
	{\homset{}{B}{C}} & {\homset{}{B}{A}}
	\arrow["{\lambda C}", from=2-1, to=1-1]
	\arrow["{\lambda A}"', from=2-2, to=1-2]
	\arrow["Fg", from=1-1, to=1-2]
	\arrow["{\homset{}{B}{g}}"', from=2-1, to=2-2]
\end{tikzcd}\]
commutes for any arrow $ g : C\to A $. Since $ C $ is any object, then let us set $ C = B $. In other words, when $ C = B $, for any arrow $ i\in \homset{}{B}{B} $, we must have $ Fg((\lambda B) i) = \lambda A (\homset{}{B}{g}(i)) $. But since $ \homset{}{B}{G}(i) = g\circ i \in  \homset{}{B}{A} $ and $ (\lambda B)(i) \in FB $, therefore this is equivalent to the condition over which we construct the map $ I $. \\\\
We now show that $ I $ is the 2-sided inverse of $ \varphi $. \\
Consider any natrual transformation $ \lambda \in \Nat{\homset{}{B}{-}}{F} $. Then,
\begin{equation*}
	\begin{split}
		I\circ \varphi(\lambda) &= I((\lambda B)(\Id{B}))
	\end{split}
\end{equation*}
But now we see that $ I((\lambda B)(\Id{B}))  = \lambda$ because $ \lambda $ satisfies the condition of $ I $ as follows:
	\begin{align*}
		Fg((\lambda B)(\Id{B})) &= Fg\circ \lambda B (\Id{B}) && \text{Note that $ (\lambda B)(\Id{B}) \in FB$}\\
		&= \lambda A \circ \homset{}{B}{g} (\Id{B})\;\forall\;A \in \obj{\cat{C}}&& \text{$ \because \;\lambda$ is a natural trans.}\\
		&= \lambda A (g\circ \Id{B})&&\\
		&= \lambda A(g)
	\end{align*}
Hence, we must have that $ I((\lambda B)(\Id{B})) = \lambda $. Which implies that
\begin{equation*}
	\begin{split}
		I \circ \varphi = \Id{\Nat{\homset{}{B}{-}}{F}}
	\end{split}
\end{equation*}
Similarly, we now evaluate the following where $ k\in FB $ is a set element:
\begin{equation*}
	\begin{split}
		\varphi\circ I (k) &= \varphi(I(k))\\
		&= (I(k)B) (\Id{B})
	\end{split}
\end{equation*}
Since $ I(k) $ is a natural transformation between $ \homset{}{B}{-} $ and $ F $ satisfying $ (I(k) A)g = Fg(k) $ for $ g : B\to A $, then, if we set $ g = \Id{B} $, then this condition reduces to
\begin{equation*}
	\begin{split}
		(I(k) B)\Id{B} &= F\Id{B}(k)\\
		&= \Id{FB} k\\
		&= k.
	\end{split}
\end{equation*}
That is, $ \varphi \circ I(k) = k $ for any $ k\in FB $. Therefore, we can write
\[\varphi \circ I  = \Id{FB}\]
Hence, we have that $ I $ is the 2-sided inverse of $ \varphi $.\\
 Therefore $ \varphi $ is an isomorphism in $ \cat{Set} $.\\\\
 \textbf{Act 2.} Now, to show $ \phi : \gamma \to \chi $ is a Natural Transformation. \\
 We need to show that the following commutes for any arrow $ f : B \to B^{\prime} $ in $ \cat{C} $ and arrow $ \lambda : F\to F^{\prime}$ in $ \Nat{\homset{}{B}{-}}{F} $ such that $ (B,F) $ and $ (B^{\prime}, F^{\prime}) $ are two objects and $ (f,\lambda) $ is an arrow in $ \cat{C}\times\Func{\cat{C}}{\cat{Set}} $:
	 \begin{equation}\label{Eq-3}
	 	\begin{tikzcd}
	 		{\gamma(B,F) = \Nat{\homset{}{B}{-}}{F}} && {\chi(B,F) = FB} \\
	 		\\
	 		{\gamma(B^\prime,F^\prime) = \Nat{\homset{}{B^\prime}{-}}{F^\prime}} && {\chi(B^\prime,F^\prime) = F^\prime B^\prime}
	 		\arrow["{\phi(B,F)}", from=1-1, to=1-3]
	 		\arrow["{\phi(B^\prime,F^\prime)}"', from=3-1, to=3-3]
	 		\arrow["{\gamma(f,\lambda)}"', from=1-1, to=3-1]
	 		\arrow["{\chi(f,\lambda)}", from=1-3, to=3-3]
	 	\end{tikzcd}
	 \end{equation}
 To prove above's commutativity, take any $ \mu \in \Nat{\homset{}{B}{-}}{F} $, then we see that
 \begin{align*}
 	\chi(f,\lambda)\circ \phi(B,F) (\mu) &=  \chi(f,\lambda)(\mu B(\Id{B}))&&\text{ By the defn. of $ \phi $}\\
 	&=F^{\prime} f \circ \lambda B (\mu B(\Id{B})) &&\text{By the defn. of $ \chi $}\\
 	&=\left ( F^\prime f \circ \lambda B \circ \mu B \right ) (\Id{B})&&\\
 	&= \left (\lambda B^{\prime} \circ Ff \circ \mu B\right )(\Id{B}) && \text{$ \because \;\lambda $ is a Nat. Trans.}\\
 	&=\left (\lambda B^{\prime} \circ \mu B^{\prime} \circ \homset{}{B}{f}\right )(\Id{B})&& \text{$ \because \;\mu$ is a Nat. Trans.}\\
 	&= \left (\lambda B^{\prime } \circ \mu B^{\prime} \right ) (\homset{}{B}{f} (\Id{B}))\\
 	&= \left (\lambda B^{\prime } \circ \mu B^{\prime} \right ) (f\circ \Id{B})&&\text{By the defn. of $ \homset{}{B}{-} $}\\
 	&= \left (\lambda B^{\prime } \circ \mu B^{\prime} \right )(f)
 \end{align*}
  Now, similarly, we can reduce the other side as follows:
  \begin{align*}
  	\phi(B^{\prime}, F^{\prime}) \circ \gamma (f,\lambda) (\mu)&= \phi(B^{\prime}, F^{\prime}) \left (\lambda \circ \mu \circ \homset{}{f}{-}\right )&& \text{By defn. of $ \gamma $}\\
  	&= \left (\left (\lambda \circ \mu \circ \homset{}{f}{-}\right )(B^{\prime})\right )(\Id{B^{\prime}})&&\text{By the defn. of $ \phi $}\\
  	&= \left (\lambda B^{\prime} \circ \mu B^{\prime} \circ \homset{}{f}{B^{\prime}}\right ) (\Id{B^{\prime}}) && \text{By the defn. of composing Nat. Trans.}\\
  	&= \left (\lambda B^{\prime} \circ \mu B^{\prime}\right ) (\homset{}{f}{B^{\prime}}(\Id{B^{\prime}}))\\
  	&= \left (\lambda B^{\prime} \circ \mu B^{\prime}\right )(\Id{B^{\prime}} \circ f)&& \text{By the defn. of $ \homset{}{-}{B^{\prime}} $}\\
  	&= \left (\lambda B^{\prime} \circ \mu B^{\prime}\right )(f)
  \end{align*}
Therefore 
\[\chi(f,\lambda)\circ \phi(B,F) = \phi(B^{\prime}, F^{\prime}) \circ \gamma (f,\lambda)\]
That is, the square in \eqref{Eq-3} commutes . \\\\
Therefore $ \phi: \gamma \to \chi $ is a Natural Transformation. But by Part 1, w proved that each component of this Natural Transformation is an Isomorphism, therefore, $ \phi $ is a Natural Equivalence between the functors $ \gamma $ and $ \chi $.
\end{proof}
\newpage

\subsection{Yoneda Embeddings}
An important corollary of the Yoneda Lemma appears when we consider $ F $ to be another hom-functor. 

\hrulefill
\begin{proposition}\label{P-3}
	(\textbf{Yoneda Embeddings}) Consider a category $ \cat{C} $. Then,
	\begin{enumerate}
		\item {The Yoneda Embedding is a \textbf{full} and \textbf{faithful} contravariant functor between $ \opcat{C} $ and $ \Func{\cat{C}}{\cat{Set}} $.}
		\item {The contravariant Yoneda Embedding, which takes an arrow $ f : A\to B$ to the natural transformation $ \homset{}{-}{A} \longrightarrow \homset{}{-}{B}$, is a \textbf{full} and \textbf{faithful} covariant functor between $ \cat{C} $ and $ \Func{\opcat{C}}{\cat{Set}} $.} 
	\end{enumerate}
\end{proposition}
\begin{proof}
	Appendix \ref{A-5} has already proved that Yoneda Embedding is a functor between $ \opcat{C} $ and $ \Func{\cat{C}}{\cat{Set}} $. Now, in the Yoneda Lemma (Proposition \ref{P-2}), if we keep $ F = \homset{}{A}{-} $ such that $ f : A\to B $ is an arrow in $ \cat{C} $, then it would result in 
	\[\Nat{\homset{}{B}{-}}{\homset{}{A}{-}} \isomorph \homset{}{A}{B}.\]
	That is, each of the arrow $ f : A\to B $ of $ \cat{C} $ can be represented by an unique (one-one correspondence) natural transformation between $ \homset{}{B}{-} $ and $ \homset{}{A}{-} $. Therefore, the functor takes each arrow in $ \homset{\cat{C}}{A}{B} $ to a unique arrow in $ \Nat{\homset{}{B}{-}}{\homset{}{A}{-}} $, making it faithful, and similarly, for every natural transform in $ \Nat{\homset{}{B}{-}}{\homset{}{A}{-}} $, there exists an arrow in $ \homset{}{A}{B} $, making the Yoneda Embedding a Full and Faithful Contravariant Functor.\\\\
	The $ 2^{\text{nd}} $ result is just the dual of the $ 1^{\text{st}} $.
\end{proof}
\hrulefill

\subsection{Representable Functors}
Another corollary of Yoneda Lemma appears when we assume that a particular natural transformation in the category $ \Nat{\homset{}{A}{-}}{F} $ is actually a Natural Equivalence, as shown now.\\\\
\point \textbf{Universal Element} : Consider a category $ \cat{C} $, an object $ A $ of $ \cat{C} $ and a functor $ F $ in $ \Func{\cat{C}}{\cat{Set}} $. If the natural transformation corresponding to set element $u\in FA $ in $ \Nat{\homset{}{A}{-}}{F} $\footnote{Due to Yoneda Lemma.} is a natural equivalence, then $ u $ is a universal element for $ F $.\\\\
\point \textbf{Representable Functor} : Consider a category $ \cat{C} $, an object $ A $ of $ \cat{C} $ and a functor $ F $ in $ \Func{\cat{C}}{\cat{Set}} $ such that $ u\in FA $ is a universal element. Then $ F $ is called a representable functor represented by $ A $.

\hrulefill
\begin{proposition}
	(\textbf{Alternate Definition for Universal Elements})\footnote{This uniqueness of universality is \textbf{extremely} important!} Consider a functor $ F : \cat{C} \longrightarrow \cat{Set} $. Then,
	\[\text{$ u\in FA $ is a universal element of $ F $} \iff \text{$ \forall\;B \in \obj{\cat{C}} $ \& $ \forall \;t\in FB $, $ \exists$ exactly one \textbf{unique} $g : A\to B \in \arr{\cat{C}} $ such that $ Fg(u) = t $.}\]
\end{proposition}
\begin{proof}
	
\end{proof}
\hrulefill

\newpage
\section{Pullbacks}
\point \textbf{Commutative Cone over a Diagram} : Consider the Diagram $ D $ in a category $ \cat{C} $ as follows
\begin{equation} \label{Diag D}
	\begin{tikzcd}
		& B \\
		A & C
		\arrow["f"', from=2-1, to=2-2]
		\arrow["g", from=1-2, to=2-2]
	\end{tikzcd}
\end{equation}
Then, the pair $ (T,x,y) $ where $ T $ is an object and $ x : T\to A$, $ y:T\to B $ are arrows in $ \cat{C} $, is called a Commutative Cone over $ D $ defined by $ T $\footnote{It is sometimes useful to call such a triple $ (T,x,y) $ an \emph{element of $ D $ defined on $ T $}.} if
\begin{equation}\label{pullback}
	\begin{tikzcd}
		T & B \\
		A & C
		\arrow["f"', from=2-1, to=2-2]
		\arrow["g", from=1-2, to=2-2]
		\arrow["y", from=1-1, to=1-2]
		\arrow["x"', from=1-1, to=2-1]
	\end{tikzcd}
\end{equation}
\textbf{commutes.} Moreover, the \textbf{set} of all commutative cones over $ D $ defined by $ T $ is dentoed by $ \Cone{T}{D} $.\\\\
\point \textbf{Contravariant $ \Cone{-}{D} $ Functor} : One can define a map 
\[\Cone{-}{D} : \opcat{C} \longrightarrow \cat{Set}\]
such that it maps
\begin{itemize}
	\item {\textbf{Object} $ K $ in $ \cat{C} $ to the set of commutative cones over $ D $ defined by $ K $, 
\[\Cone{-}{D}(K) = \Cone{K}{D}.\]	
}
	\item {\textbf{Arrow} $ h : T \to P $ in $ \cat{C} $ to a set function between $ \Cone{P}{D}  \to \Cone{T}{D}$ defined by
\[\Cone{-}{D}(h)(P,x,y) = \Cone{h}{D}(P,x,y) = (T,x\circ h, y\circ h).\]	
}
\end{itemize}
This map then defines a contravariant functor between $ \cat{C} $ and $ \cat{Set} $.\\\\
\point \textbf{Pullback} : Consider the Diagram D in \eqref{Diag D}. A commutative cone $ (P,p_1,p_2) $ over $ D $ is said to be a Pullback of Diagram $ D $\footnote{Also called the \emph{Fiber Product} of Diagram $ D $.}, if 
\[\text{$ (P,p_1,p_2) $ is a Universal Element for $ \Cone{-}{D} $.}\]
Equivalently, expanding the definition of Universal Element means : If $ (P,p_1,p_2) $ is a pullback of diagram $ D $, then
\begin{enumerate}
	\item {the diagram \[\begin{tikzcd}
			P & B \\
			A & C
			\arrow["f"', from=2-1, to=2-2]
			\arrow["g", from=1-2, to=2-2]
			\arrow["{p_2}", from=1-1, to=1-2]
			\arrow["{p_1}"', from=1-1, to=2-1]
		\end{tikzcd}\]
		\textbf{commutes} and}
	\item {for any object $ T $ in $ \cat{C} $ and for any commutative cone over $ D $ defined on $ T $, $ (T,x,y) $, there exists an unique arrow $ h : T \to P $ in $ \cat{C} $ such that the diagram
	\begin{equation}\label{Diag pull}
		\begin{tikzcd}
			T \\
			& P & B && {} \\
			& A & C
			\arrow["f"', from=3-2, to=3-3]
			\arrow["g", from=2-3, to=3-3]
			\arrow["{p_2}"', from=2-2, to=2-3]
			\arrow["{p_1}", from=2-2, to=3-2]
			\arrow["h"{description},dashed, from=1-1, to=2-2]
			\arrow["x"', from=1-1, to=3-2]
			\arrow["y", from=1-1, to=2-3]
		\end{tikzcd}
	\end{equation}
\textbf{commutes}\footnote{Hence, the set of all commutative cones over $ D $ defined by $ T $ are in one-to-one correspondence to the $ \homset{}{T}{P} $.}. The Figure \eqref{Diag pull} is then called a \textbf{\textit{Pullback Diagram}}.
}
\end{enumerate} 
\newpage

\hrulefill
\begin{proposition}
	Consider the Pullback 
	\[\begin{tikzcd}
		P & B \\
		A & C
		\arrow["{p_2}", from=1-1, to=1-2]
		\arrow["{p_1}"', from=1-1, to=2-1]
		\arrow["f"', from=2-1, to=2-2]
		\arrow["g", from=1-2, to=2-2]
	\end{tikzcd} \]
of Diagram $ D $. Then, 
\[\begin{tikzcd}
	T & B \\
	A & C
	\arrow["y", from=1-1, to=1-2]
	\arrow["x"', from=1-1, to=2-1]
	\arrow["f"', from=2-1, to=2-2]
	\arrow["g", from=1-2, to=2-2]
\end{tikzcd}\;\text{ is a Pullback of \eqref{Diag D}} \iff \exists ! \;h : T \overset{\sim}{\to} P \text{ which is an isomorphism such that }	\begin{tikzcd}
T \\
& P & B\\
& A & C
\arrow["f"', from=3-2, to=3-3]
\arrow["g", from=2-3, to=3-3]
\arrow["{p_2}"', from=2-2, to=2-3]
\arrow["{p_1}", from=2-2, to=3-2]
\arrow["h"{description},dashed, from=1-1, to=2-2]
\arrow["x"', from=1-1, to=3-2]
\arrow["y", from=1-1, to=2-3]
\end{tikzcd}\text{ commutes.}\]
\end{proposition}
\begin{proof}
	\textbf{L $ \implies $ R} : Given that  $ (P,p_1,p_2) $ and $ (T,x,y) $ are pullbacks. Therefore there exists an unique arrows $h : T\to P $ and $ k : P\to T $ such that 
\[\begin{tikzcd}
	T \\
	& P & B \\
	& A & C
	\arrow["{p_1}", from=2-2, to=3-2]
	\arrow["f"', from=3-2, to=3-3]
	\arrow["g"', from=2-3, to=3-3]
	\arrow["{p_2}"', from=2-2, to=2-3]
	\arrow["h"{description}, shift right=1, dashed, from=1-1, to=2-2]
	\arrow["x"', curve={height=6pt}, from=1-1, to=3-2]
	\arrow["y", curve={height=-6pt}, from=1-1, to=2-3]
	\arrow["k"{description}, shift left=2, dashed, tail reversed, no head, from=1-1, to=2-2]
\end{tikzcd}\]
commutes. Now consider the arrow $ h\circ k $. Since $ h$ \& $ k $ are unique, therefore $ h \circ k $ is also unique. This means that 
\begin{align*}
	h\circ k \circ x &= h\circ (k\circ x)\\
	&= h\circ p_1\\
	&= x
\end{align*}
Therefore, $ h\circ k \circ x = x $, but $ \Id{T} \circ x  = x$ too, but we also have that $ h\circ k$ is unique, therefore $ h\circ k = \Id{T} $. Similarly, $ k\circ h = \Id{P} $. Therefore, $ h $ and $ k $ forms a unique isomorphism between $ T $ and $ P $. \\\\
\textbf{R $ \implies  $ L} : If $ h : T \overset{\sim}{\to} P $ is an unique isomorphism and $ (P,p_1,p_2) $ is a pullback, then for any other commutative cone $ (F,m,n) $ of $ D $, there exists a unique arrow $ k : F \to P $ such that it's corresponding Pullback diagram commutes. But since $ P $ and $ T $ are isomorphic, therefore there exists an unique arrow, formed by composition of $ k $, such that it's corresponding pullback diagram over $ (T,x,y) $ also commutes. Since $ (F,m,n) $ is any such commutative cone over $ D $, therefore $ (T,x,y) $ is also a Pullback over $ D $.
\end{proof}

\hrulefill
\newpage
\section{Limits}
Last section we saw usage of term \emph{Diagram $ D $}, but we didn't formally defined what a diagram means. In the process of it's definition, we would explore the concept of limits in a category, which arises from graph theoretic viewpoint of them.
\subsection{Graphs}
\point \textbf{Graph} : A graph $ G $ consists of two sets and two functions on those sets:
\begin{itemize}
	\item {\textbf{Object} set $ O $ of nodes of $ G $,}
	\item {\textbf{Arrow} set $ A $ of edges of $ G $, and}
	\item {Two \textbf{functions} $ d_0,d_1 : A \to O $ such that 
\begin{equation*}
	\begin{split}
		d_0(f) &= \text{Source node of edge $ f $}\\
		d_1(f) &= \text{Target node of edge $ f $.}
	\end{split}
\end{equation*}	
Therefore, a graph $ G $ can be defined as a \emph{Category without composition}.
}
\end{itemize}
\point \textbf{Graph Homomorphism} : A function $ F : G\to H$ between two graphs $ G $ and $ H $ is called a graph homomorphism if it maps objects to objects, arrows to arrows such that for any edge $ f : A\to B $ in graph $ G $, $ F $ would maps $ f $ to 
\[Ff : FA \to FB\]
where $ FA , FB$ are nodes in $ H $ and $ Ff $ an edge in $ H $.\\\\
\point \textbf{Underlying Graph} : Consider any category $ \cat{C} $. Clearly, there is an underlying graph corresponding to $ \cat{C} $, formed by collection of objects and arrows. This graph is called underlying graph, denoted $ \ulg{\cat{C}} $.
\begin{itemize}
	\item {Moreover, any functor $ F : \cat{C} \to \cat{D} $ between categories therefore forms a graph homomorphism $ \ulg{F} : \ulg{\cat{C}} \to \ulg{\cat{D}}$.}
	\item {Hence, this provides an \textbf{underlying graph functor} $ Y $:
\[Y : \cat{Cat} \longrightarrow \cat{Graph}\]
between category of categories and functors to category of graph and homomorphisms.	
}
\end{itemize}
\subsection{Diagrams}
\point \textbf{Diagram} : A diagram in a category $ \cat{C} $ is a graph homomorphism 
\[D : I \to \ulg{\cat{C}}\]
where $ I $ is some graph called \emph{Index Graph}. The diagram $ D $ is generally referred to as the \emph{Diagram $ D $ of type $ I $}. 
\begin{itemize}
	\item {Equivalently, we can refer diagram $ D $ as a functor from an index category $ \cat{I} $ to $ \cat{C} $ so that we write
\[D : \cat{I} \to \cat{C}.\]	}
\item {Therefore any diagram of type $ \cat{I} $ is an object in the functor category $ \Func{\cat{I}}{\cat{C}} $.}
	\item {$ D $ is called a finite diagram if both the object and arrows sets of index graph are finite.
	}
\end{itemize}
\point \textbf{Natural Transformation of Diagrams} : Consider two diagrams $ D, E  : I \to \cat{C}$. A natural transformation 
\[\lambda : D \to E\]
between them is defined as the collection of all arrows
\[\lambda k : Dk \to Ek \]
where $ k $ is any node/object of $ I $ such that the following commutes for any edge $ b : i \to j $ in $ I $:
\[\begin{tikzcd}
	Di & Ei \\
	Dj & Ej
	\arrow["{\lambda i}", from=1-1, to=1-2]
	\arrow["Eb", from=1-2, to=2-2]
	\arrow["{\lambda j}"', from=2-1, to=2-2]
	\arrow["Db"', from=1-1, to=2-1]
\end{tikzcd}\]
\newpage
\subsection{Limits}
\point \textbf{General Commutative Cones} : Consider a category $ \cat{C} $ and a diagram $ D : \cat{I} \to \cat{C} $, which is just a functor. Moreover, consider a constant functor $ E: \cat{I} \to \cat{C} $ such that it takes any object of $ \cat{I} $ to a constant $ W $ in $ \cat{C} $. A \textbf{commutative cone with vertex $ W $} on the diagram $ D $ is then defined as a natural transformation 
\[\alpha : E\longrightarrow D.\]
This is generally denoted as 
\[cone\;\alpha : W\to D .\]
\begin{itemize}
	\item {Therefore, the commutative cone $ cone\;\alpha : W\to D $ is an arrow in \textbf{category of diagrams of type $ \cat{I} $}. This, in the notion of elements, can be said to be \emph{the cone $ \alpha $ is an element of diagram $ D $ defined over $ W $}, justifying the prior use of it.} 
	\item {The \textbf{components} of the natural transformation $ \alpha $, $ \alpha i $ for object $ i $ in $ \cat{I} $, are then the arrows in $ \cat{C} $(!)}
	\item {The naturality condition furthermore demands that if $cone\; \alpha :  W\to D$ is a commutative cone with vertex $ W $ over a diagram $ D $, then for \textbf{any} arrow $ e : i\to j $ in $ \cat{I} $, the following must commute:
\[\begin{tikzcd}
	& W \\
	Di && Dj
	\arrow["De"', from=2-1, to=2-3]
	\arrow["{\alpha i}"', from=1-2, to=2-1]
	\arrow["{\alpha j}", from=1-2, to=2-3]
\end{tikzcd}\]	
Note that the naturality square here is just a \emph{triangle} because the functor $ E $ maps any object to constant object $ W $ and any arrow to $ \Id{W} $.
}
\end{itemize}
\point \textbf{Contravariant $ \Cone{-}{D} $ functor} : In the same spirit as earlier, we define cone functor as a map 
\[\Cone{-}{D} : \opcat{C} \longrightarrow \cat{Set}\]
such that it maps
\begin{itemize}
	\item {\textbf{Object} $ W $ of $ \cat{C} $ to the set of all commutative cones with vertex $ W $ over a diagram $ D $, $ \Cone{W}{D} $. Note that this set is just the collection of all natural transformations between the constant functor to $ W $ and the diagram functor of type $ \cat{I} $.}
	\item {\textbf{Arrow} $ h : V\to W $ of $ \cat{C} $ to the set function
\[\Cone{h}{D} : \Cone{W}{D}\longrightarrow\Cone{V}{D}\]
	such that for any natural transformation $ \alpha \in \Cone{W}{D} $,
	\[\Cone{h}{D}(\alpha) = \beta\]
	where $ \beta $ is the commutative cone with vertex $ V $ whose collection of arrows is $ (\alpha i) \circ h $ for all objects $ i $ of $ \cat{I} $.}
\end{itemize}
\point \textbf{Limit} : Consider a diagram $ D $ of type $ \cat{I} $ in a category $ \cat{C} $. Then, the limit of $ D $, $ \lim D $, is defined as:
\[\text{$ \lim D $ is a Universal Element of $ \Cone{-}{D} $.}\]
\footnote{Note that the object representing the functor $ \Cone{-}{D} $ and the universal element of $ \Cone{-}{D} $ are sometimes used interchangeably when referring to the limit $ \lim D $.}Equivalently, we can say : if for \textbf{any} commutative cone $ \alpha $ with any vertex $ A $, there exists an arrow $ h : A\to W $ in $ \cat{C} $ such that $ \Cone{h}{D}(\lim D) =  \alpha$, then $ \lim D $ is called the limit of the diagram $ D $.

\hrulefill
\begin{proposition}\label{P-6}
	Consider a diagram $ D: \cat{I} \to \cat{C} $ and let two commutative cones $ \alpha : W \to D $ and $ \beta : V \to D $ both be limits of $ D $. Then, there exists a unique isomorphism between them, $ u : V\to W $, such that
	\[(\alpha i)\circ u = \beta i\;\text{for all objects }i\text{ of }  \cat{I} .  \]
\end{proposition}
\begin{proof}
Take any two limits $ \alpha : W\to D $ and $ \beta :V\to D $ of a diagram $ D : \cat{I} \to \cat{C}$. Since $ \alpha \;\& \;\beta $ are limits, therefore they are universal elements of $ \Cone{-}{D} $. This means that there exists a \textbf{unique} arrow $ f : V\to W $ and $ g : W\to V $ such that $ \Cone{f}{D}(\alpha) = \{(\alpha i) \circ f \;:\; i \in \obj{\cat{I}}\} = \beta $ and $ \Cone{g}{D}(\beta) = \{(\beta i) \circ g\;:\;i\in \obj{\cat{I}}\} = \alpha $. Hence, we have that $ \beta i = (\alpha i) \circ f $ and $ \alpha i = (\beta i) \circ g $. Combining them gives that $ \alpha i = (\alpha i) \circ (f\circ g) $. But the same is true for $ \Id{W} $ as $ \alpha i = (\alpha i) \circ \Id{W} $. Therefore, by uniqueness of $ f\circ g $, we must have that $ f\circ g = \Id{W} $. Similarly $ g\circ  f = \Id{V} $, proving that $ f $ is an isomorphism. Therefore, there exists an isomorphism between two limits of a diagram $ D $.
\end{proof}
\newpage
\point \textbf{Complete Categories\footnote{Also called \textbf{Left Exact Categories}.}} : A category $ \cat{C} $ is called complete if \emph{every} diagram in $ \cat{C} $ has a limit. Moreover, $ \cat{C} $ is called \textbf{finitely complete} if every \emph{finite} diagram in $ \cat{C} $ has a limit.

\subsection{Products}
\point \textbf{Product of objects} : Consider a category $ \cat{C} $ and two arbitrary objects $ A $ and $ B $ in it. The limit of the \emph{arrow-less} or \emph{discrete} diagram $ (A,B) $ is called the product of $ A $ and $ B $ and is thus denoted by
\[ cone \;\alpha : A\times B \to D.\]
That is, the product $ A \times B$ is nothing but the collection of pairs of elements of $ A $ and $ B $.
\begin{itemize}
	\item {It is hence clear that the product $ A\times B$ contains two \textbf{projection} arrows $ p_1 : A\times B \to A $ and $ p_2 : A\times B \to B $ such that for any $ h = (x,y) \in A\times B $,
\begin{align*}
	p_1(h) &= x\\
	p_2(h) &= y
\end{align*}	
}
\item {Moreover, since $ A \times B $ is the vertex of the limit of diagram $ (A,B) $, therefore more explicitly we can write the definition of product $ A \times B$ as the object in $ \cat{C} $ with the property that for \textbf{any} other commutative cone over any vertex $ W $ of diagram $ (A,B) $ with corresponding arrows $ f $ and $ g $, there exists a \textbf{unique} arrow denoted $ (f,g)$ such that the following \textbf{commutes}:
\[\begin{tikzcd}
	& {A\times B} \\
	A && B \\
	& W
	\arrow["{(f,g)}"{description}, dashed, from=3-2, to=1-2]
	\arrow["{p_1}"', from=1-2, to=2-1]
	\arrow["{p_2}", from=1-2, to=2-3]
	\arrow["f", from=3-2, to=2-1]
	\arrow["g"', from=3-2, to=2-3]
\end{tikzcd}\]
where product of arrows is defined more generally as follows:}
\end{itemize}
\point \textbf{Product of arrows} : Product of $ f : A\to C $ and $ g : B\to D $ is defined as 
\[f\times g = (f\circ p_1, g\circ p_2) \; : \; A\times B \longrightarrow C\times D.\]
\begin{itemize}
	\item {Hence, for any member $ (x,y) \in A\times B$, we have
\[(f\times g) (x,y) = (f(x),g(y)).\]	
}
\end{itemize}
\point \textbf{Product of Indexed Set of Objects} : The product of an indexed collection can be defined by extension of the above as following: Consider the indexed set of objects to be a discrete category $ \cat{I} $ without any arrows. The product of the objects in $ \cat{I} $ is then defined as the limit of the diagram $ D : \cat{I} \to \cat{C} $. This product is generally denotes as\footnote{Sometimes, for better representation, it is useful to write product as $ \prod D_i $.}
\[\prod_{i\in \obj{\cat{I}}} Di.\]
In a more explicit manner, the product of some indexed set of objects, regarded as discrete category $ \cat{I} $, is the object $ \prod_{i\in \cat{I}} Di$ in category $ \cat{C} $ where $ D : \cat{I}  \to \cat{C}$ is the discrete diagram, such that for any other commutative cone with any vertex $ W $ of $ \cat{C} $ over discrete diagram $ D $, there exists a unique arrow $ u : W \to \prod Di $ such that the following commutes:
\[\begin{tikzcd}
	&& {\prod D_i} \\
	\\
	{D_1} & {D_2} & \dots \\
	&& W
	\arrow["{p_1}"{description}, from=1-3, to=3-1]
	\arrow["{p_2}"{description}, from=1-3, to=3-2]
	\arrow["\dots"{description}, draw=none, from=1-3, to=3-3]
	\arrow["{f_1}"{description}, from=4-3, to=3-1]
	\arrow["{f_2}"{description}, from=4-3, to=3-2]
	\arrow["{u = \prod f_i}"{description}, curve={height=12pt}, dotted, from=4-3, to=1-3]
\end{tikzcd}\]
\newpage
\point \textbf{Category with Products} : A category $ \cat{C} $ is said to have products if for \textbf{any} (possibly infinite) indexed set of objects $ \cat{I} $ and for the corresponding discrete diagram $ D : \cat{I} \to \cat{C} $, the product $ \prod D_i$ exists. Equivalently, if the limit of $ D  $ exists.
\begin{itemize}
	\item {If the product exists only for any finite set of objects, then the category $ \cat{C} $ is said to have only \textbf{finite products}.}
\end{itemize}
\point \textbf{Left Exact} or \textbf{Flat Functors} : Consider a functor $ F : \cat{C} \longrightarrow\cat{D}$ where $ \cat{C} $ and $ \cat{D} $ are finitely complete categories. Then, $ F $ is called a Left Exact or a Flat Functor if it preserves finite limits.\\\\
\point \textbf{Continuous Functor} : A functor $ F : \cat{C} \to \cat{D} $ is called continuous if for every small category diagram $ D : \cat{I} \to \cat{C} $, the functor $ F $ preserves it's limits.
\subsection{Equalizers, Equivalence Relations \& Kernel Pairs}
\point \textbf{Equalizer} : The equalizer of two arrows $ f,g: A\to B $ in a category $ \cat{C} $ is the limit of the diagram :
\[\begin{tikzcd}
	A && B
	\arrow["f"', shift right=2, from=1-1, to=1-3]
	\arrow["g", shift left=2, from=1-1, to=1-3]
\end{tikzcd}\]
and is generally denoted as $ \Eq{f}{g} $. Therefore, equalizer represents the pair of elements $ (x,y) $ where $ x\in A $ and $ y\in B $ such that $ f(x) = g(x) = y $.

\hrulefill
\begin{proposition}\label{P-7}
	(\textbf{Projections of a Product are jointly monic}) Given two parallel arrows $ f, g : A\to B\times C $,
	\[f = g\;\iff\; p_1\circ f = p_1 \circ g \;\&\;p_2 \circ f = p_2 \circ g.\]
\end{proposition}
\begin{proof}
	L $ \implies $ R, is trivial to see. Suppose $ p_1 \circ f = p_1 \circ g : A\to B$ and $ p_2 \circ f = p_2 \circ g : A\to C$. Remember that $ B\times C $ is the limit object of the discrete diagram $ B, C $. Hence, there exists a unique arrow $ h : A\to B\times C $ such that $ p_1\circ h =  p_1 \circ f = p_1 \circ g$ and $ p_2 \circ h  = p_2 \circ f = p_2 \circ g$. Since $ h $ is unique with this property, therefore $ f=g=h$.
\end{proof}
\begin{remark}
	Therefore, kernel pairs always define a subobject of the product. Here, by subobject, we mean the source of the monomorphism which defines the isomorphism class of the subobject.
\end{remark}
\hrulefill

\newpage
\point \textbf{Internal Equivalence Relations}\label{IER} : An equivalence relation on an object $ X $ of a finitely complete category $ \cat{C} $ is a subobject $ R $ of the product $ X\times X $ identified by the joint monomorphism arrow\footnote{See Proposition \ref{P-7}.} $ (s,t) : R\rightarrowtail X\times X $ such that it is further equipped with three specific arrows as follows:
\begin{enumerate}
	\item {\textbf{Reflexivity} : A unique arrow $\rho : X\to R $ such that 
	\[s\circ \rho = \Id{X}\;\&\;t\circ \rho = \Id{X}.\] }
\item {\textbf{Symmetry} : A unique arrow $ \sigma : R\to R $ such that 
\[s\circ \sigma = t\;\&\;t\circ \sigma = s.\]}
\item {\textbf{Transitivity} : A unique arrow $ \tau : R\times_X R \to R $ whose source is the pullback $ (R\times_X R, \tilde{s}, \tilde{t}) $ of $ t $ along $ s $ such that 
\[s\circ \tau = s\circ \tilde{s} \;\&\; t\circ \tau = t\circ \tilde{t}.\]
}
\end{enumerate}
These three arrows $ \rho, \sigma, \tau $ can be represented as the following commutative diagrams:
% https://q.uiver.app/?q=WzAsMjUsWzIsMiwiUiJdLFsyLDQsIlgiXSxbNCw0LCJZIl0sWzQsMiwiWCJdLFszLDMsIlhcXHRpbWVzIFgiXSxbMCwwLCJYIl0sWzgsMiwiUiJdLFsxMCwyLCJYIl0sWzgsNCwiWCJdLFsxMCw0LCJZIl0sWzksMywiWFxcdGltZXMgWCJdLFs2LDAsIlIiXSxbNSwxMSwiUiJdLFs1LDEzLCJYIl0sWzcsMTEsIlgiXSxbNywxMywiWSJdLFs2LDEyLCJYXFx0aW1lcyBYIl0sWzMsOSwiUlxcdGltZXNfWFIiXSxbNSw2LCJSXFx0aW1lc19YUiJdLFs1LDgsIlIiXSxbNyw4LCJYIl0sWzcsNiwiUiJdLFszLDUsIlxccmhvLVxcdGV4dHtSZWZsZXhpdml0eX0iXSxbOSw1LCJcXHNpZ21hIC1cXHRleHR7U3ltbWV0cnl9Il0sWzYsMTQsIlxcdGF1LVxcdGV4dHtUcmFuc2l0aXZpdHl9Il0sWzAsMywidCJdLFswLDEsInMiLDJdLFsxLDIsImYiLDJdLFszLDIsImYiXSxbMCw0LCIocyx0KSIsMSx7InN0eWxlIjp7InRhaWwiOnsibmFtZSI6Im1vbm8ifX19XSxbNCwxLCJwXzEiXSxbNCwzLCJwXzIiLDJdLFs1LDAsIlxccmhvIiwxLHsic3R5bGUiOnsiYm9keSI6eyJuYW1lIjoiZGFzaGVkIn19fV0sWzUsMywiXFx0ZXh0e0lkfV9YIiwxXSxbNSwxLCJcXHRleHR7SWR9X1giLDFdLFs2LDgsInMiLDJdLFs2LDcsInQiXSxbNiwxMCwiKHMsdCkiLDEseyJzdHlsZSI6eyJ0YWlsIjp7Im5hbWUiOiJtb25vIn19fV0sWzgsOSwiZiIsMl0sWzcsOSwiZiJdLFsxMCw4LCJwXzEiXSxbMTAsNywicF8yIiwyXSxbMTEsNiwiXFxzaWdtYSIsMSx7InN0eWxlIjp7ImJvZHkiOnsibmFtZSI6ImRhc2hlZCJ9fX1dLFsxMSw4LCJ0IiwxXSxbMTEsNywicyIsMV0sWzEyLDEzLCJzIiwyXSxbMTIsMTQsInQiXSxbMTMsMTUsImYiLDJdLFsxNCwxNSwiZiJdLFsxNiwxMywicF8xIl0sWzE2LDE0LCJwXzIiLDJdLFsxMiwxNiwiKHMsdCkiLDEseyJzdHlsZSI6eyJ0YWlsIjp7Im5hbWUiOiJtb25vIn19fV0sWzE3LDEyLCJcXHRhdSIsMSx7InN0eWxlIjp7ImJvZHkiOnsibmFtZSI6ImRhc2hlZCJ9fX1dLFsxNywxMywic1xcY2lyYyBcXHRpbGRle3N9IiwxXSxbMTcsMTQsInRcXGNpcmMgXFx0aWxkZXt0fSIsMV0sWzIxLDIwLCJzIl0sWzE5LDIwLCJ0IiwyXSxbMTgsMjEsIlxcdGlsZGV7dH0iXSxbMTgsMTksIlxcdGlsZGV7c30iLDJdLFsxOCwyMCwiIiwxLHsic3R5bGUiOnsibmFtZSI6ImNvcm5lci1pbnZlcnNlIn19XV0=
\[\begin{tikzcd}
	X &&&&&& R \\
	\\
	&& R && X &&&& R && X \\
	&&& {X\times X} &&&&&& {X\times X} \\
	&& X && Y &&&& X && Y \\
	&&& {\rho-\text{Reflexivity}} &&&&&& {\sigma -\text{Symmetry}} \\
	&&&&& {R\times_XR} && R \\
	\\
	&&&&& R && X \\
	&&& {R\times_XR} \\
	\\
	&&&&& R && X \\
	&&&&&& {X\times X} \\
	&&&&& X && Y \\
	&&&&&& {\tau-\text{Transitivity}}
	\arrow["t", from=3-3, to=3-5]
	\arrow["s"', from=3-3, to=5-3]
	\arrow["f"', from=5-3, to=5-5]
	\arrow["f", from=3-5, to=5-5]
	\arrow["{(s,t)}"{description}, tail, from=3-3, to=4-4]
	\arrow["{p_1}", from=4-4, to=5-3]
	\arrow["{p_2}"', from=4-4, to=3-5]
	\arrow["\rho"{description}, dashed, from=1-1, to=3-3]
	\arrow["{\text{Id}_X}"{description}, from=1-1, to=3-5]
	\arrow["{\text{Id}_X}"{description}, from=1-1, to=5-3]
	\arrow["s"', from=3-9, to=5-9]
	\arrow["t", from=3-9, to=3-11]
	\arrow["{(s,t)}"{description}, tail, from=3-9, to=4-10]
	\arrow["f"', from=5-9, to=5-11]
	\arrow["f", from=3-11, to=5-11]
	\arrow["{p_1}", from=4-10, to=5-9]
	\arrow["{p_2}"', from=4-10, to=3-11]
	\arrow["\sigma"{description}, dashed, from=1-7, to=3-9]
	\arrow["t"{description}, from=1-7, to=5-9]
	\arrow["s"{description}, from=1-7, to=3-11]
	\arrow["s"', from=12-6, to=14-6]
	\arrow["t", from=12-6, to=12-8]
	\arrow["f"', from=14-6, to=14-8]
	\arrow["f", from=12-8, to=14-8]
	\arrow["{p_1}", from=13-7, to=14-6]
	\arrow["{p_2}"', from=13-7, to=12-8]
	\arrow["{(s,t)}"{description}, tail, from=12-6, to=13-7]
	\arrow["\tau"{description}, dashed, from=10-4, to=12-6]
	\arrow["{s\circ \tilde{s}}"{description}, from=10-4, to=14-6]
	\arrow["{t\circ \tilde{t}}"{description}, from=10-4, to=12-8]
	\arrow["s", from=7-8, to=9-8]
	\arrow["t"', from=9-6, to=9-8]
	\arrow["{\tilde{t}}", from=7-6, to=7-8]
	\arrow["{\tilde{s}}"', from=7-6, to=9-6]
	\arrow["\ulcorner"{anchor=center, pos=0.125}, draw=none, from=7-6, to=9-8]
\end{tikzcd}\]
\emph{Appendix \ref{A-7} shows the result of this notion of Equivalence Relation in the $ \cat{Set} $.} 
\newpage
\point \textbf{Kernel Pair} : A kernel pair for an arrow $ f : A\to B $ in a category $ \cat{C} $ is the pullback of $ f $ along itself. That is, in the following, $ s,t $ is the kernel pair of $ f $:
\[\begin{tikzcd}
	E && A \\
	\\
	A && B
	\arrow["f", from=1-3, to=3-3]
	\arrow["f"', from=3-1, to=3-3]
	\arrow["s"', from=1-1, to=3-1]
	\arrow["t", from=1-1, to=1-3]
	\arrow["\ulcorner"{anchor=center, pos=0.125}, draw=none, from=1-1, to=3-3]
\end{tikzcd}\]
\begin{remark}
	In $ \cat{Set} $, an equivalence relation $ (u,v) $ is the kernel pair of the class map.
\end{remark}
\subsection{Existence of Limits (Theorem)}
The following theorem states the equivalent conditions for the existence of limits in a category.

\hrulefill
\begin{theorem} 
	In any category $ \cat{C} $, the following are equivalent:
	\begin{enumerate}
		\item {$ \cat{C} $ has all \textbf{Finite Limits}.}
		\item {$ \cat{C} $ has :
	\begin{enumerate}
		\item {a \emph{\textbf{Terminal Object}},}
		\item {all \emph{\textbf{Equalizers}} of parallel pairs, and}
		\item {all \emph{\textbf{Binary Products}}.}
	\end{enumerate}	
	}
\item {$ \cat{C} $ has:
	\begin{enumerate}	
		\item {a \emph{\textbf{Terminal Object}}, and}
		\item {all \emph{\textbf{Pullbacks}}.}
	\end{enumerate}
	}
	\end{enumerate}
This further means that a category $ \cat{C} $ has \emph{\textbf{all limits}} if and only if it has \emph{\textbf{all equalizers}} of parallel pairs and \emph{\textbf{all products}}.
\end{theorem}
\begin{proof}
	\textbf{1 $ \implies $ 2 and 1 $ \implies $ 3}. Suppose $ \cat{C} $ has all finite limits. Then since terminal object is the limit of empty diagram, therefore it exists in $ \cat{C} $. Since Equalizer is the limit of any pair of parallel arrows:
	\[\begin{tikzcd}
		A & B
		\arrow["f"', shift right=1, from=1-1, to=1-2]
		\arrow["g", shift left=1, from=1-1, to=1-2]
	\end{tikzcd}\]
therefore it must exist too for any such pair. Finally, Binary Products are just the object representing the limit of the discrete diagram $ X , X $. \\\\
\textbf{3 $ \implies  $ 2}. Suppose $ \cat{C} $ has a Terminal object and all Pullbacks. To see that all binary products exists, consider $ T $ to be the terminal object, therefore the pullback of the diagram $  A \overset{t_A}{\rightarrow} T \overset{t_B}{\leftarrow}B$ :
\[\begin{tikzcd}
	P & B \\
	A & T
	\arrow["{t_B}", from=1-2, to=2-2]
	\arrow["{t_A}"', from=2-1, to=2-2]
	\arrow["u"', from=1-1, to=2-1]
	\arrow["v", from=1-1, to=1-2]
	\arrow["\ulcorner"{anchor=center, pos=0.125}, draw=none, from=1-1, to=2-2]
\end{tikzcd}\]
Since $ P $ is universal with this property (it's a Pullback), therefore $ P $ is an object equipped with two arrows $ u$ and $ v $ such that for any other object $ R $ and arrows $ a : R\to A $ and $ b : R\to B $, there exists an unique arrow $ h : R\to P $ such that $ u\circ h = a $ and $ v\circ h = b $. This is exactly what product $ A\times B $ would require. \\
For showing that all Equalizers for any parallel pair $ f,g :A\to B $ exists, similarly consider the following pullback diagram (note that we now use the fact that products exists, as just proved):
\[\begin{tikzcd}
	E \\
	& P & A \\
	& B & {B\times B} \\
	& B && B
	\arrow["{(\Id{B},\Id{B})}", from=3-2, to=3-3]
	\arrow["{(f,g)}", from=2-3, to=3-3]
	\arrow["s", from=2-2, to=3-2]
	\arrow["t"', from=2-2, to=2-3]
	\arrow["b"{description}, from=1-1, to=3-2]
	\arrow["a"{description}, from=1-1, to=2-3]
	\arrow["h"{description}, dashed, from=1-1, to=2-2]
	\arrow["{p_1}"{description}, from=3-3, to=4-2]
	\arrow["{p_2}"{description}, from=3-3, to=4-4]
\end{tikzcd}\]
which thus implies that $ (\Id{B},\Id{B})\circ s = (f,g)\circ t $, which on simplification yields that $ s = f\circ t = g\circ t $. Since $ P $ is universal with this property, therefore it is just the $ \Eq{f}{g} $. Hence, if $ \cat{C} $ has Pullbacks and a Terminal object, then it further has all Binary Products and all Equalizers.
\\\\
$ \dagger $ \textbf{2 $ \implies  $ 1}. Suppose a category $ \cat{C} $ satisfies the property 2. Terminal object is present because it is present(!) Now, due to existence of terminal object and binary products, all finite products exists by induction (base step uses terminal object). Now, take any non-empty finite diagram $ D : \cat{I} \to \cat{C} $. Since all finite products exists, therefore both
\begin{align*}
	A &= \prod_{i \in \obj{\cat{I}}} Di \\
	B &= \prod_{a \in \arr{\cat{I}}} D \tar{a}
\end{align*}
exists. Now, since all Equalizers exists, therefore, construct the following two parallel arrows $ f $ and $ g $, so that the following commutes:
\[\begin{tikzcd}
	{\prod_{a \in \arr{\cat{I}}} D \tar{a}} && {\prod_{i \in \obj{\cat{I}}} Di} && {\prod_{a \in \arr{\cat{I}}} D \tar{a}} && {\prod_{i \in \obj{\cat{I}}} Di} \\
	\\
	{D\tar{a}} && {D\src{a}} &&& {D\tar{a}}
	\arrow["{p^{B}_{\tar{a}}}"', from=1-1, to=3-1]
	\arrow["{p^A_{\src{a}}}", from=1-3, to=3-3]
	\arrow["Da", from=3-3, to=3-1]
	\arrow["f"', from=1-3, to=1-1]
	\arrow["{p^B_{\tar{a}}}"', from=1-5, to=3-6]
	\arrow["{p^A_{\tar{a}}}", from=1-7, to=3-6]
	\arrow["g"', from=1-7, to=1-5]
\end{tikzcd}\]
Note that such $ f $ and $ g $ exists because of universality of the product objects $ A $ and $ B $ and that all objects in the diagram $ D $ which are targets of arrows of the diagram are contained in the set of all objects of the diagram. Now, we can argue about the equalizer of $ f $ and $ g $:
\[\begin{tikzcd}
	{\prod_{i \in \obj{\cat{I}}} Di} && {\prod_{a \in \arr{\cat{I}}} D \tar{a}}
	\arrow["f", shift left=2, from=1-1, to=1-3]
	\arrow["g"', shift right=2, from=1-1, to=1-3]
\end{tikzcd}\]
The $ \Eq{f}{g} $ would be the object such that
\[\begin{tikzcd}
	{\Eq{f}{g}} \\
	\\
	{\prod_{i \in \obj{\cat{I}}} Di} && {\prod_{a \in \arr{\cat{I}}} D \tar{a}}
	\arrow["f", shift left=2, from=3-1, to=3-3]
	\arrow["g"', shift right=2, from=3-1, to=3-3]
	\arrow["s"', from=1-1, to=3-1]
	\arrow["t", from=1-1, to=3-3]
\end{tikzcd}\]
commutes. This means that $ f\circ s = g\circ s = t $. But this means that by the above diagrams, the map $ s:\Eq{f}{g} \longrightarrow D$ is a commutative cone. Moreover, since $ \Eq{f}{g} $ is universal with this property, therefore, it is the limit of the diagram $ D $\footnote{Note that this proof didn't needed to be this long. Vaguely speaking, since we wanted to prove that for any diagram a limit of it exists, therefore it is natural to construct such an object first. We then see that since everything has to commute for such a limit, then it is natural to construct a product object of all such objects for which an arrow ends on it. The rest becomes trivial.}.
\end{proof}
\hrulefill
\newpage
\subsection{Preservation of Limits (Theorem)}
\point \textbf{Limit Preserving Functor} : Consider a functor $ F : \cat{C} \to \cat{D} $ and a diagram in $ \cat{C} $ of type $ \cat{I} $, $ D : \cat{I} \to \cat{C} $. Consider the limit of $ D $ as the cone:
\[cone\;\alpha : \lim D \to D\]
which is the universal element of $ \Cone{-}{D} $. Then, the functor $ F $ is said to be preserving the $ \lim D $ if 
\[cone \; F\alpha : F\lim D \to FD\]
is the universal element of $ \Cone{-}{FD} $.
\begin{remark}
	Note that in the above definition:
	\begin{enumerate}
		\item {Since $ \alpha $ is a natural transformation between constant functor to $ \lim D $ and the $ D $, therefore $ F\alpha $ is the natural transformation $ F(\alpha i) $ for all $ i \in \obj{\cat{I}} $. }
		\item {The $ FD $ is supposed to be the composition of functors so that the diagram $ FD $ is in $ \cat{D} $.}
	\end{enumerate}
\end{remark}
\hrulefill
\begin{theorem}
	Consider a functor $ F : \cat{C} \to \cat{D} $ and a diagram $ D : \cat{I} \to \cat{C} $ and $ cone\;\alpha : \lim D \to D $ is the universal cone of $ D $. Then, $ F $ preserves the limit of $ D $ \textbf{if and only if} 
	\begin{enumerate}
		\item {$ FD $ has a limit 
	\[cone\;\beta : \lim FD \longrightarrow FD.\]	
	and,
	}
\item {$ \exists $ an isomorphism $ g : F(\lim D) \longrightarrow \lim FD $ such that for any object $ T $ of $ \cat{C} $, the following \textbf{commutes}:
\[\begin{tikzcd}
	{\homset{\cat{D}}{FT}{F(\lim D)}} \\
	\\
	{\homset{\cat{D}}{FT}{\lim FD}} && {\Cone{FT}{FD}}
	\arrow["{\homset{\cat{D}}{FT}{g}}"', from=1-1, to=3-1]
	\arrow["{\homset{\cat{D}}{FT}{\beta}}"', from=3-1, to=3-3]
	\arrow["{\homset{\cat{D}}{FT}{F\alpha}}"{description}, from=1-1, to=3-3]
\end{tikzcd}\]
}
	\end{enumerate}
\end{theorem}
\begin{proof}
	\textbf{L $\implies$ R.} Suppose that the function $ F : \cat{C}  \to \cat{D}$ preserves limit of diagram $ D $ and any arrow $ h : T\to \lim D $. This means that $ F(\lim D) $ is the limit of $ FD $. Moreover, we have that $ \alpha \circ h $ is a cone of $ D $, therefore, $ F(\alpha \circ h) = F\alpha \circ Fh $ is a cone for diagram $FD$. By Proposition \ref{P-6}, there exists an isomorphism $ g : F(\lim D) \to \lim FD $ such that $ F(\alpha \circ h) = \beta \circ g \circ Fh $, which is just the commutative diagram in 2.\\\\
	\textbf{R $ \implies $ L.} Suppose that $ FD $ has a limit and there exists an isomorphism $ g : F(\lim D) \to \lim FD $ such that for any arrow $ f : FT\to F(\lim D) $, 
	\[ F\alpha = \beta \circ g \implies F\alpha \circ \inv{g} = \beta .\]
	With this, we wish to show that $ F $ preserves the limit of $ D $, in other words, $ F(\lim D) $ is a limit of $ FD $. Clearly, $ F(\lim D) $ can be a vertex for a commutative cone over $ FD $, due to the $ F\alpha $. Now to see it's universality, take any commutative cone over $ FD $ with vertex, say $ M $, denoted by $ \gamma $. Then, by universality of $ \lim FD $, there exists a unique arrow $ m : M\to \lim FD $ such that $ \beta \circ m = \gamma $. But now, $ \inv{g} \circ m : M \to F(\lim D) $ is also unique. Moreover, we have that $\gamma = \beta \circ m = F\alpha \circ \inv{g} \circ m $. Hence, for any commutative cone over $ FD $, $ F(\lim D) $ acts as a universal element for $ \Cone{-}{FD} $, therefore $ F $ preserves limits.
\end{proof}
\hrulefill
\subsection{Creation \& Reflection of Limits}
\point \textbf{Limit Creating Functor} : Suppose $ F : \cat{C} \to \cat{D} $ is a functor and $ D : \cat{I} \to \cat{C} $ is a diagram of type $ \cat{I} $. Moreover, suppose that the limit of diagram $ FD $ in $ \cat{D} $ exists and the limiting cone is:
\[cone\;\beta : \lim FD \longrightarrow FD.\]
Then, $ F $ is said to be creating limits of type $ \cat{I} $ if there exists a \textbf{unique} limiting cone 
\[cone\;\alpha : X \to D\]
in $ \cat{C} $ such that $ F\alpha = \beta $.
\newpage
\point \textbf{Limit Reflecting Functor} : Suppose $ F : \cat{C} \to \cat{D} $ is a functor and $ D:  \cat{I} \to \cat{C} $ is a diagram of type $ \cat{I} $. Moreover, suppose that $ FD $ has the following limiting cone:
\[cone \;\beta : \lim FD \to FD\]
and there exists a cone 
\[cone \;\alpha : X\to D\]
in $ \cat{C} $ such that $ F\alpha = \beta $. Then, $ F $ is said to reflect the limits of type $ \cat{I} $ if $ \alpha $ is the limiting cone of $ D $.

\hrulefill

\emph{Typeset the answers to the questions solved in the diary when the time permits.}
\newpage
\section{Duality}
Most of the concepts defined earlier in these notes actually have a dual concept. We now state them in full detail.\\\\
\point \textbf{Dual Concepts} : Consider a category $ \cat{C} $. A dual concept is then the same concept but in opposite category $ \opcat{C} $. Some of such concepts are:
\begin{enumerate}
	\item {\textbf{Commutative Cocone} : Suppose a diagram $ D: \cat{I} \to \cat{C} $. A commutative cocone \emph{from} the diagram $ D $ to a vertex $ W $ is a natural transformation 
\[cocone\;\alpha : D\to W.\]	
Note that the set of commutative cocones with vertex $ W $ from a diagram $ D $ is denotes as
\[\Cocone{D}{W}.\]
Hence, $ \Cocone{D}{-} : \cat{C} \to \cat{Set} $ is a covariant functor by composition.
} 
\item {\textbf{Colimit} : Suppose a diagram $ D : \cat{I} \to \cat{C} $ is given. A colimit of $ D $ is then 
\[\text{Colimit of $ D $ is the Universal Element of $ \Cocone{D}{-} $.}\]
Expanding this, it means that for any object $ W $ of $ \cat{C} $ and any commutatuve cocone $ \alpha \in \Cocone{D}{W}$, there exists a unique arrow $ f : \colim D \to W $ such that $ \Cocone{D}{f}(\beta) = \alpha  $ where $ \beta : D \to \colim D$ is the colimiting cocone and $ \colim D $ is the vertex of this cocone $ \beta $. 
} 
\item {\textbf{Sum} or \textbf{Coproduct}: Consider a discrete diagram $ (A,B) $ in $ \cat{C} $ where $ A $ and $ B $ are any of it's objects. Then, the coproduct of $ A $ and $ B $ is given by the colimit of the discrete diagram $ (A,B) $ and as such is denoted by either $ A+B $ or $ A \coprod B $.\\\\
Expanding the definition, coproduct $ A\amalg B $ is the vertex of the colimiting cocone over the discrete diagram $ (A,B) $. That is, for any object $ W $ which has arrows $ f : A \to W $ and $ g : B \to W $, there exists a unique arrow $ \ang{f,g} : A\amalg B \to W $ such that $ \ang{f,g}\circ i_1 = f$ and $ \ang{f,g}\circ i_2 = g $ where $ i_1 : A \to A\amalg B$ and $ i_2 : B \to A\amalg B $ are the two \textbf{injection arrows} for $ A\amalg B $. That is, the following commutes:
\[\begin{tikzcd}
	& {A\amalg B} \\
	A && B \\
	& W
	\arrow["{i_1}", from=2-1, to=1-2]
	\arrow["{i_2}"', from=2-3, to=1-2]
	\arrow["f"', from=2-1, to=3-2]
	\arrow["g", from=2-3, to=3-2]
	\arrow["{\langle f,g\rangle}"{description}, dashed, from=1-2, to=3-2]
\end{tikzcd}\]}
\item {\textbf{Coequalizer} : Consider two parallel arrows $ f,g : A\to B $ in a category $ \cat{C} $. The Coequalizer is then defined as the colimit of the diagram:
\[\begin{tikzcd}
	A && B
	\arrow["g"', shift right=2, from=1-1, to=1-3]
	\arrow["f", shift left=2, from=1-1, to=1-3]
\end{tikzcd}\]
Again, expanding the definition, this means that coequalizer of $ f $
 and $ g $ is the arrow $ h : B\to W $ such that 
\begin{itemize}
	\item {$ h\circ f = h\circ g $, }
	\item {For any object $ X $ which has arrows $ s: B\to X $ with the property that $ s\circ f = s\circ g $, then there exists a unique arrow $ k : W \to X $ such that $ s = k\circ h $.}
\end{itemize}
} 
\item {\textbf{Pushout} : This is the dual concept to pullbacks. Consider the following diagram in any category $ \cat{C} $:
\[\begin{tikzcd}
	W \\
	\\
	&& P && B \\
	\\
	&& A && C
	\arrow["g"', from=5-5, to=3-5]
	\arrow["f", from=5-5, to=5-3]
	\arrow["{i_1}", from=5-3, to=3-3]
	\arrow["{i_2}"', from=3-5, to=3-3]
	\arrow["s", from=5-3, to=1-1]
	\arrow["t"', from=3-5, to=1-1]
	\arrow["h"{description}, dashed, from=3-3, to=1-1]
\end{tikzcd}\]
The commutation of this for any such object $ W $ makes $ P $ the pushout of $ A\overset{f}{\leftarrow }C \overset{g}{\rightarrow } B $.
} 
\item {\textbf{Preservation} : A functor which preserves finite colimits is called a \textbf{right-exact} functor. Similarly, a functor is \textbf{cocontinuous} if it preserves colimits of all small diagrams.}
\end{enumerate}
\subsection{Regularity}
\begin{proposition}\label{P-8}
	Every Equalizer is a Monomorphism.
\end{proposition}
\begin{proof}
	Take any parallel pair of arrows $ f,g : A\to B $. Then take it's Equalizer, denoted as $ e : E \to A$, which would thus have the property that $ f\circ e = g\circ e $ and $ E $ is universal with this property. Hence, take any other object $ W $ which has two arrows $ s,t : W \to E $. Therefore, suppose $ e\circ s = e\circ t : W \to A $ are two arrows, for which we can see they satisfies:
	\begin{align*}
		f\circ (e\circ s) &= (f\circ e)\circ s\\
		&= (g\circ e)\circ s\\
		f\circ (e\circ t) &= (f\circ e)\circ t\\
		&= (g\circ e)\circ t
	\end{align*}
Therefore, there must exist a unique arrow $ p : W\to E $ such that $ e\circ p = e\circ s = e\circ t$. But $ s, t $ are already two arrows with this property, therefore, it must be that $ s= t =p $. Hence, $ e\circ s = e\circ t $ implies $ s = t $, so, $ e $ is a monomorphism.
\end{proof}
\hrulefill

\point \textbf{Regular Monomorphism} : The Equalizer of two parallel arrows, which is a monomorphism (Proposition \ref{P-8}), is called a regular monomorphism.\\\\
\point \textbf{Regular Epimorphism} : The Coequalizer of two parallel arrows, which is a epimorphism (Dual of Proposition \ref{P-8}), is called a regular epimorphism.\\\\
\point \textbf{Regular Category} : A category $ \cat{C} $ is called regular if it satisfies:
\begin{enumerate}
	\item {Every finite diagram has a \textbf{Limit},}
	\item {Every parallel pair of arrows has a \textbf{Coequalizer},}
	\item {$ \dagger $ For any pullback square
\[\begin{tikzcd}
	A & B \\
	C & D
	\arrow["f", from=1-1, to=1-2]
	\arrow["h", from=1-2, to=2-2]
	\arrow["k"', from=2-1, to=2-2]
	\arrow["g"', from=1-1, to=2-1]
\end{tikzcd}\]
where $ h : B\to D $ is a\textbf{ regular epimorphism}, then it's pullback $ g: A \to C $ must also be a \textbf{regular epimorphism}.	
}
\end{enumerate} 
\subsubsection{A striking result : Arrow Factorization Theorem in Regular Categories}
\begin{theorem}
	In a regular category $ \cat{C} $, \textbf{\emph{any arrow}} $ f $ can be written as
	\[f = m\circ e\]
	where $ m $ is a \textbf{monomorphism} and $ e $ is a \textbf{regular epimorphism}.
\end{theorem}
\begin{proof}
\emph{Proof in Diary, $ 7^{th} $ April, 2020.}
\end{proof}
\hrulefill
\subsection{Exact Categories}
\point \textbf{Effective/Congruent Equivalence Relation} : An Equivalence Relation \ref{IER} $ (s,t) : R \rightarrowtail X\times X$ is called Effective or Congruent if $ s $ and $ t $ forms a kernel pair for some arrow $ f : X\to Y $ in $ \cat{C} $.\\\\
\point \textbf{Exact Category} : A category $ \cat{C} $ is called exact if it is regular and every equivalence relation is a congruence. That is, each equivalence relation is a kernel pair for some arrow.
\begin{proposition}
	We have the following two results:
	\begin{itemize}
		\item {If $ \cat{C} $ is a \textbf{Regular} category, then for any object $ A $, the slice category $ \cat{C}/A $ is also \textbf{Regular}.}
		\item {If $ \cat{C} $ is an \textbf{Exact} category, then for any object $ A $, the slice category $ \cat{C}/A $ is also \textbf{Exact}.}
	\end{itemize}
\end{proposition}
\hrulefill
\newpage
\section{Adjoint Functors}
\point \textbf{Hom-functor with two arguments} : In a category $ \cat{C} $, a hom-functor in two arguments can be defined as a functor between:
\[\homset{\cat{C}}{-}{-} : \opcat{C}\times \cat{C} \longrightarrow \cat{Set}\]
where $ \homset{\cat{C}}{-}{-}(A,B) = \homset{\cat{C}}{A}{B} $ for objects $ A $ and $ B $ of $ \cat{C} $ and $ \homset{\cat{C}}{-}{-}(f,g) = \homset{\cat{C}}{f}{g} $ for $ f: A\to B $ and $ g: C\to D $ which is the following set function:
\begin{align*}
	\homset{\cat{C}}{f}{g} &: \homset{\cat{C}}{B}{C}\to \homset{\cat{C}}{A}{D}\\
	&\text{where, for any $ h:B\to C $,}\\
	&\homset{\cat{C}}{f}{g}(h) = g\circ h\circ f : A\to D
\end{align*}
\point \textbf{Adjoint Functors} : Suppose $ \cat{C} $ and $ \cat{D} $ are two given categories with $ L:\cat{C}\longrightarrow\cat{D} $ and $ R:\cat{D}\longrightarrow\cat{C} $ be two functors between them. Then, one defines $ L $ to be the Left Adjoint of $ R $ and $ R $ to be the Right Adjoint of $ L $ if there exists a \textbf{natural isomorphism} 
\begin{align*}
	\alpha :   \homset{\cat{D}}{L(-)}{-} \overset{\isomorph}{\longrightarrow} \homset{\cat{C}}{-}{R(-)}.
\end{align*}
The naturality condition means that for any arrows $ f : A\to B $ in $ \cat{C} $ and $ g : C\to D $ in $ \cat{D} $, the following natural square \textbf{commutes}:
\[\begin{tikzcd}
	{\homset{\cat{C}}{LB}{C}} && {\homset{\cat{D}}{B}{RC}} \\
	\\
	{\homset{\cat{C}}{LA}{D}} && {\homset{\cat{D}}{A}{RD}}
	\arrow["{\homset{\cat{C}}{Lf}{g}}"', from=1-1, to=3-1]
	\arrow["{\homset{\cat{D}}{f}{Rg}}", from=1-3, to=3-3]
	\arrow["{\alpha(B,C)}", from=1-1, to=1-3]
	\arrow["{\alpha(A,D)}"', from=3-1, to=3-3]
	\arrow["\isomorph"', from=1-1, to=1-3]
	\arrow["\isomorph", from=3-1, to=3-3]
\end{tikzcd}\]
One denotes that $ L $ and $ R $ are adjoint by the following:
\[\begin{tikzcd}
	{\cat{C}} && {\cat{D}}
	\arrow[""{name=0, anchor=center, inner sep=0}, "R", shift left=2, from=1-3, to=1-1]
	\arrow[""{name=1, anchor=center, inner sep=0}, "L", shift left=2, from=1-1, to=1-3]
	\arrow["\dashv"{anchor=center, rotate=-90}, draw=none, from=1, to=0]
\end{tikzcd}\]
\textbf{\emph{Remark}} : Suppose $ L $ and $ R $ are the given two adjoint functors, then, since for any object $ A $ of $ \cat{C} $ and $ B $ of $ \cat{D} $, the sets $ \homset{\cat{C}}{A}{RB} $ and $ \homset{\cat{D}}{LA}{B} $ are isomorphic, therefore the arrow $ g : LA \to B $ in $ \homset{\cat{D}}{LA}{B} $, corresponding to $ f : A\to RB $ in $ \homset{\cat{C}}{A}{RB} $, is called the \textbf{adjoint transpose} of $ f $. That is, $ g $ is called adjoint transpose of $ f $, and vice-versa.\\\\
\point \textbf{Unit \& Counit of Adjoint} : Suppose $ L : \cat{C}\to \cat{D} $ and $ R : \cat{D} \to\cat{C}$ are adjoint functors. Then, for any object $ A$ in $ \cat{C} $, we would have
\[\homset{\cat{C}}{A}{RLA} \overset{\isomorph}{\longrightarrow} \homset{\cat{D}}{LA}{LA}.\]
Therefore, the adjoint transpose of $ \Id{LA} : LA\to LA  $ in $ \homset{\cat{D}}{LA}{LA} $, which would be, say, $ \bar{\Id{LA}} : A\to RLA $, is called the \textbf{unit of the adjunction at object $ A $} of $ \cat{C} $. Perhaps, one can write the\textbf{\emph{ \underline{unit as a natural transformation}}} $ \eta : \Id{\cat{C}}\longrightarrow R\circ L$ so that for any object $ A $ of $ \cat{C} $, $ \eta A : A \to RLA $.\\\\
Similarly, the adjoint transpose of $ \Id{RB} : RB \to RB$ in $ \homset{\cat{C}}{LRB}{B} $ is called the \textbf{counit of the adjunction at object $ B $} of $ \cat{D} $. Similar to above, \textbf{\emph{\underline{counit is the natural transformation}}} $ \epsilon : L\circ R \longrightarrow \Id{\cat{D}} $.
\newpage
\subsection{Results on Adjoint Functors}
\textbf{\emph{Remark}}. Suppose $ L: \cat{C}\longrightarrow \cat{D}$ and $ R : \cat{D}\to \cat{C} $ are adjoint functors. Hence, for any object $ C $ of $ \cat{C} $ we would have a natural bijection of the sort
\[ \alpha_2 : \homset{\cat{D}}{LC}{-} \overset{\isomorph}{\longrightarrow} \homset{\cat{C}}{C}{R(-)}.\]
By Yoneda Lemma, we must therefore have a universal element for the functor $ \homset{\cat{C}}{C}{R(-)} $, in the set $ \homset{\cat{C}}{C}{RLC} $, to which the above natural bijection corresponds to. Hence, $ LC $ would be the object representing $ \homset{\cat{C}}{C}{R(-)} $ and the corresponding universal element is just the unit evaluated at $ C $, $ \eta C : C \to RLC $.

\subsubsection{Point-wise Construction of Adjoints}
\hrulefill
\begin{theorem}
	Consider $ \cat{C} $ and $ \cat{D} $ are two categories.
	\begin{enumerate}
		\item {If $ R : \cat{D} \longrightarrow \cat{C}$ is a functor, then: 
	\[\text{$\forall \;C \in \obj{\cat{C}}\;,\;\; \homset{\cat{C}}{C}{R(-)} $ is \textbf{Representable} $ \implies $ $ R $ has a \textbf{Left Adjoint}.}\]	
	}
\item {If $ L : \cat{C}\longrightarrow\cat{D} $ is a functor, then:
\[\text{$ \forall\; D \in \obj{\cat{D}} $, $ \homset{\cat{D}}{L(-)}{D} $ is \textbf{Representable} $ \implies $ $ L $ has a \textbf{Right Adjoint}.}\]
}
	\end{enumerate}
\end{theorem}
\begin{proof}
	\emph{Written in Diary at $ 4^{th} $ March, 2020.}
\end{proof}
\hrulefill
\subsubsection{Adjoints preserves (co)limit}
\begin{theorem}\label{T-5}
	Suppose $ L : \cat{C}\longrightarrow \cat{D} $ and $ R : \cat{D} \longrightarrow \cat{C} $ are Adjoint Functors. Then,
	\begin{enumerate}
		\item{$ R $ \textbf{preserves limit} of any finite diagram in $ \cat{D} $.}
		\item{$ L $ \textbf{preserves colimit} of any finite diagram in $ \cat{C} $.}
	\end{enumerate}
\end{theorem}
\begin{proof}
	\emph{Written in Diary at $ 29^{th} $ February, 2020.}
\end{proof}
\hrulefill
\subsubsection{$ \bigstar $ Freyd's Adjoint Functor Theorem}
\point \textbf{Solution Set Condition} : Suppose $ R : \cat{D} \longrightarrow\cat{C} $ is a functor. Then $ R $ is said to follow the Solution Set Condition if $ \forall \;C\in \obj{\cat{C}}$, $ \exists $ a set $ S_C $ called the solution set of $ C $
\[S_C = \{(y,B)\;\vert\; y : C\to RB\;,\;\;B\in \obj{\cat{D}}\}\]
such that for any arrow $ z : C\to RD $, $ \exists \;(y,B) \in S_C$ and $ f: B\to D \in \arr{\cat{D}}$ for which the following commutes:
\[\begin{tikzcd}
	C & RB \\
	& RD
	\arrow["y", from=1-1, to=1-2]
	\arrow["z"', from=1-1, to=2-2]
	\arrow["Rf", from=1-2, to=2-2]
\end{tikzcd}\] 
\point \textbf{Weak Universal Arrow} : Suppose $ R : \cat{D} \longrightarrow\cat{C}$ is a right adjoint functor. Also suppose that the solution set of $ R $ for any object $ C $ is simply a singleton, given by $ S_C =  \{(y,B)\}$. With this, one will see that the arrow $ y : C \to RB $ acts as a universal element of $ \homset{\cat{C}}{C}{R(-)} $ but without the uniqueness condition. Hence, $ y $ is called a weak universal arrow for $ R $ and $ C $.

\hrulefill
\begin{theorem}
	(\textbf{Freyd's Adjoint Functor Theorem}) Suppose $ \cat{D} $ is a category with all limits. Then,
	\begin{align*}
		\text{Functor $ R : \cat{D} \longrightarrow \cat{C} $ has a \textbf{Left Adjoint} } \bm{\iff} &\text{ $ \bm{1.} $ $ R $ preserves all \textbf{Limits}, and}\\
		 &\text{ $ \bm{2.} $ $ R $ satisfies the\textbf{ Solution Set Condition}}
	\end{align*}
\end{theorem}
\begin{proof}
	\emph{Discussed in Appendix \ref{A-8}.}
\end{proof}
\hrulefill
 	















\newpage
\appendix\section{Interesting Proofs}
\subsection{Hom Functors.}
\label{PROOF-1}\begin{proof}
	First note that for any arrow $ f\in \arr{\cat{C}} : B\to C$, we have $ \homset{\cat{C}}{A}{f} : \homset{\cat{C}}{A}{B} \to \homset{\cat{C}}{A}{C} $ which is clearly an arrow of category of sets $ \cat{Set} $. Now, take an object $ B\in  \obj{\cat{C}}$, then we have $ \Id{B} \in\arr{\cat{C}} $ is such that $ \homset{\cat{C}}{A}{\Id{B}} : \homset{\cat{C}}{A}{B} \to \homset{\cat{C}}{A}{B} $ which is $ \Id{\homset{\cat{C}}{A}{B}} $. Finally, take $ f:B\to C $ and $ g:C\to D $ be two arrows in $ \arr{\cat{C}} $. Also take any $ h\in \homset{\cat{C}}{A}{B} $. Then, we will have the following from the definition,
	\begin{equation*}
		\begin{split}
			\homset{\cat{C}}{A}{g\circ f}(h) &= g \circ f\circ h\\
			\homset{\cat{C}}{A}{g} \circ \homset{\cat{C}}{A}{f} (h) &= \homset{\cat{C}}{A}{g} (f\circ h)\\
			&= g\circ f\circ h.
		\end{split}
	\end{equation*}
	Hence $ \homset{\cat{C}}{A}{g\circ f} = \homset{\cat{C}}{A}{g}\circ \homset{\cat{C}}{A}{f} $, completing the proof.
\end{proof}
\subsection{Category of Categories \& Functors, $ \cat{Cat} $.}
\label{PROOF-2}\begin{proof}
	Consider the category of categories $ \cat{Cat} $ with functors as arrows between them. Take any $ F \in \arr{\cat{Cat}} $. Clearly, $ F $ has a source category $ \cat{C} $ and target category $ \cat{D} $. Take any $ \cat{C},\cat{D} \in \obj{\cat{Cat}}$ and consider $ F $ to be a functor between them in $ \arr{\cat{Cat}} $. Now take another functor $ G : \cat{B}\to \cat{C} $. We need to show that $ F \circ G $ is also an arrow of $ \cat{Cat} $. For this, we first need to prove that $ F\circ G $ is actually a functor. This is easy to see as mapping arrows to arrows and preservation of identity is simple, however, for preserving composition, consider composable $ f,g\in \arr{\cat{B}} $, therefore, $ f\circ g \in \arr{\cat{B}} $. Now, $ G(f\circ g) = Gf \circ Gg $ and then $ F(G(f\circ g))  = F(Gf\circ Gg) = FGf\circ FGg$, hence $ F\circ G $ is also a functor. Now, take any arrow $ f\in \arr{\cat{B}} $. We have $ Gf \in \arr{\cat{C}} $ and then $ FGf \in \arr{\cat{D}} $. Therefore, $ F\circ G (f) \in \arr{\cat{D}} $ or in other words, $ F\circ G : \cat{B} \to \cat{C} $, hence $ F\circ G \in \arr{\cat{Cat}} $. We also trivially have the identity functor $ F:\cat{A} \to \cat{A} $ in $ \arr{\cat{Cat}} $. Associativity of Functors also follows from the definition of their composition. Finally, any identity functor $ \Id{\cat{C}} $ and any functor $ F : \cat{C} \to \cat{D}$ in $ \arr{\cat{Cat}} $ are such that $ F\circ \Id{\cat{C}} = F$ 
	s, for any $ f\in \arr{\cat{C}} $, we have $ F\circ \Id{\cat{C(f)}} = F(\Id{\cat{C}}(f)) = F(f) $.
\end{proof}
\subsection{Determinant Map is a Natural Transformation.}
\label{PROOF-3} \begin{proof}
	The precise statement for the fact that determinant is a natural transformation is as follows :\\
Let $ \GL{n} $ denote the functor which takes a ring with identity to an $ n\times n $ invertible matrix with entries from that ring:
\[\GL{n} : \cat{Ring} \longrightarrow \cat{Group}.\]
This functor $ \GL{n} $ maps objects and arrows of $ \cat{Ring} $ as follows:
\begin{itemize}
	\item {$ \GL{n}(R) = M_n $, where $ R\in \obj{\cat{Ring}} $ and $ M_n\in \obj{\cat{Group}} $. Note that $ R $ is a ring with identity and $ M_n $ is the group of $ n\times n $ invertible matrices with entries from $ R $.}
	\item {$ \GL{n}(f:R\to S) = \GL{n}f : \GL{n}(R) \to \GL{n}(S) \equiv M_n^{R} \to M_n^{S} $, where $ \GL{n}f $ is an entry-wise mapping from between matrix groups $ M_n^{R} \to M_n^{S} $ of $ f $. That is, $ \GL{n} f(m_{ij}^{R}) = m_{ij}^{S}$.}
\end{itemize}
This is the precise definition of $ \GL{n} $ functor. Now, the next functor to introduce is the Units functor, $ \Un $.\\
Define the Group of Units functor $ \Un $ as follows:
\[\Un : \cat{Ring} \longrightarrow \cat{Group}\]
such that 
\begin{itemize}
	\item {$ \Un(R) = R^{\times} $ where $ R^{\times} $ is the group of units of ring $ R $\footnote{A unit of ring $ R $ is an element $ r\in R $ such that $ \inv{r}\in R$. Since $ (R,\cdot) $ is already a monoid ($ R $ is a ring with identity), therefore collection of units of $ R $ with $ (\cdot ) $ forms a group $ R^{\times} $, called group of units of $ R $.}.}
	\item {For an arrow $ f: R\to  S$ in $ \arr{\cat{Ring}} $, functor $ \Un $ maps it to
\[\Un f : \Un(R) \longrightarrow \Un(S) \equiv R^{\times} \longrightarrow S^{\times}\]
	such that $ \Un f(r) = f(r) \in S^{\times} \;\forall\;r\in R^{\times}$. Note that $ R^{\times} \subseteq R $ and $ S^{\times} \subseteq S $. 
}
\end{itemize}
Now, we need to show that 
\[\lambda : \GL{n} \longrightarrow \Un\]
which maps a matrix to it's determinant is a natural transformation.\\
This means that $ \lambda $ is the collection of arrows, one for each ring $ R $ in $ \cat{Ring} $$ R $ in $ \cat{Ring} $,
\[\lambda R : \GL{n}R = M_n^{R} \longrightarrow \Un R= R^{\times}\]
maps each matrix in $ M_n^{R} $ to it's determinant in $ R^{\times} $. Now consider any ring homomorphism $ g : R\to S $ in $ \cat{Ring} $. Note that for any matrix $ M\in M_n^{R} $, $ \lambda R(M) = \det(M) \in R^{\times} $ and then $ \Un f(\det(M)) = f(\det(M)) $. Similarly, $ \GL{n}f(M) = f(M) \in M_n^{S} $ element-wise and then $ \lambda S (f(M)) = \det(f(M)) $. But we know that $ f $ is a ring homomorphism, therefore $ f(\det(M)) = \det(f(M)) $. This means that
\[\begin{tikzcd}
	 \GL{n}R \arrow[swap]{d}{\lambda R}\arrow{r}{\GL{n}f} &\GL{n}S\arrow{d}{\lambda S}\\
	 \Un R \arrow[swap]{r}{\Un f} &\Un S
\end{tikzcd}\;\;\text{\textbf{commutes.}}\]
Hence $ \lambda : \GL{n} \longrightarrow \Un $ is a natural transformation.
\end{proof}
\subsection{Clarity on Subobjects}
\label{A-4}
The subobjects of an object $ A \in \obj{\cat{C}}$ are defined as follows:
\begin{enumerate}
	\item {Consider the following situation. We have two monomorphisms $ u : S\rightarrowtail A $ and $ v : T\rightarrowtail A $ such that $ u $ factors through $ v $. This means that $ u = v\circ j $ for a map $ j : S\to T $.}\\
	\item {Now, define the following relation $ \sim $ on the set of all monomorphisms with target $ A $:
\[\text{\emph{$ u\sim v$\textbf{ if and only if }$ \exists\;  $ an isomorphism $ j : S\to T $}.}\]	
}
\item {We then find that this relation $ \sim $ is actually an equivalence relation:
\begin{itemize}
	\item {$ u\sim  u $ : We have $ \Id{S} $ as the isomorphism $ \Id{S} : S\to S $.}
	\item {$ u\sim v \implies v\sim u $ : Since $ u\sim v $, therefore $ \exists  \;$ an isomorphism $ j : S\to T $, hence, the inverse of the isomorphism, denoted by $ j^{\prime} $ is also an isomorphism from $ T\to S $.}
	\item {$ u\sim v \;\&\; v\sim w\implies u\sim w$ where $ w: U\to A $ : We have two isomorphisms $ j_u : S\to T $ and $ j_v : T\to U $. Their composition would be another arrow $ j_v \circ j_u : S\to U$. Since composition of isomorphisms is an isomorphism, therefore $ u\sim w $ due to $ j_v\circ j_u $. }
\end{itemize}
Hence $ \sim $ is an equivalence relation(!)
}
\item {Due to this equivalence relation $ \sim $ on the set of all monomorphisms with target $ A $, we can hence conclude that \emph{the set of all monomorphisms with target $ A $ is partitioned into equivalence classes(!)}}
\item {In this partitioned set, any equivalence class is called a subobject of $ A $. That is, a collection of monomorphisms with target $ A $ is a subobject of $ A $ if each of the monomorphisms have source which is isomorphic to the sources of all other monomorphisms in the given subobject.\\\\
Hence, each monomorphism to an object $ A $ in a category defines a subobject of $ A $.}
\item {\emph{Example.} In the following diagram,
\[\begin{tikzcd}
	A && B \\
	&&&& Z \\
	D && C
	\arrow["{f_B}"{description}, tail, from=1-3, to=2-5]
	\arrow["{f_C}"{description}, tail, from=3-3, to=2-5]
	\arrow["{f_A}", curve={height=-30pt}, tail, from=1-1, to=2-5]
	\arrow["{f_D}"', curve={height=30pt}, tail, from=3-1, to=2-5]
	\arrow[tail reversed, from=1-1, to=1-3]
	\arrow[tail reversed, from=1-3, to=3-3]
	\arrow[tail reversed, from=3-3, to=3-1]
	\arrow[tail reversed, from=1-1, to=3-1]
	\arrow[tail reversed, from=3-3, to=1-1]
	\arrow[tail reversed, from=3-1, to=1-3]
\end{tikzcd}\]
the collection of monomorphisms $ f_A, f_B, f_C,f_D $ forms a subobject of $ Z $.
}
\end{enumerate}
\subsubsection{Subobjects as Limits of $ g : A\to B $}
It's not difficult to see that \textbf{the}\footnote{All limits of a given diagram are isomorphic, Proposition \ref{P-6}.} limit of the diagram $ g: A\to B $ is the commutative cone with vertex $ A $ with components $ \Id{A} $ and $ g $ itself, that is:
\[\begin{tikzcd}
	& A \\
	A & {} & B
	\arrow["g"', from=2-1, to=2-3]
	\arrow["{\Id{A}}"', from=1-2, to=2-1]
	\arrow["g", from=1-2, to=2-3]
\end{tikzcd}\]
commutes, to make the natural transformation associated with $ A $ the limit of $ g : A\to B $. Now, if it so happens that $ g : A\rightarrowtail  B$ is a monomorphism, then, due to universality of the limit, any other commutative cone acting as the limit of $ g : A\rightarrowtail B $ would be isomorphic to $ A $ (Proposition \ref{P-6}). That is, for any other limit on vertex $ W $, we would have an isomorphism $ f : W\to A $. Therefore all the monomorphisms with this property would have to be the limit of $ g : A\rightarrowtail B $, which is just the definition of subobject.
\subsection{Yoneda Embedding is a Functor.}
\label{A-5}
\begin{proof}
	We need to show that the following Yoneda map
	\[\lambda : \opcat{C} \longrightarrow \Func{\cat{C}}{\cat{Set}}\]
	is a contravariant functor, where each arrow $ f : A\to B $ is mapped to  $ \lambda f : \homset{}{B}{-} \to \homset{}{A}{-} $ where each component of the $ \lambda f $ at object $ C $ is given by $ \lambda fC : \homset{}{B}{C}\to \homset{}{A}{C} $ which in turn is given by $ \lambda fC g = g\circ f $ for $ g\in \homset{}{B}{C} $.\\
	To see that $ \lambda f  $ is a natural transformation (to confirm that an arrow is mapped to an arrow), simply note that for any arrow $ h : C\to C^{\prime} $, we have the following for $ a\in \homset{}{B}{C} $ :
	\[\homset{}{A}{h}(\lambda fC a) = h\circ a\circ f = \lambda f C^{\prime} \left (\homset{}{B}{h}(a)\right )\]
	so that 
	\[\begin{tikzcd}
		\homset{}{B}{C}\arrow[swap]{d}{\homset{}{B}{h}} \arrow{r}{\lambda f C} &\homset{}{A}{C}\arrow{d}{\homset{}{A}{h}}\\
		\homset{}{B}{C^{\prime}}\arrow[swap]{r}{\lambda f C^{\prime}} &\homset{}{A}{C^{\prime}}
	\end{tikzcd}\;\text{commutes,}\]
and so $ \lambda f $ is a natural transformation between hom-functors $ \homset{}{B}{-} $ and $ \homset{}{A}{-} $, so that $ \lambda f $ is an arrow in $ \Func{\cat{C}}{\cat{Set}} $. \\
Next, for any identity arrow $ \Id{B} : B\to B $, we see that $ \lambda \Id{B} : \homset{}{B}{-} \longrightarrow \homset{}{B}{-} $ is a natural transformation which takes an object $ C $ to $ \lambda \Id{B} C : \homset{}{B}{C}  \to \homset{}{B}{C}$ where $ g\in  \homset{}{B}{C} $ is taken to $ \lambda \Id{B}Cg = g\circ \Id{B} = g $, so $ \lambda \Id{B} $ is an identity natural transformation.\\
Finally, consider two arrows $ f : A\to B $ and $ g : B\to C $ so that $ g\circ f : A\to C $ in $ \cat{C} $ are mapped by $ \lambda $ to the following natural transformations (we show the component at an object $ D $):
\begin{equation*}
	\begin{split}
		\lambda g\circ f D&: \homset{}{C}{D} \to \homset{}{A}{D}\;\text{such that }\lambda g\circ f D k = k\circ (g\circ f) \;\text{for } k\in \homset{}{C}{D}\\
			\lambda  f D&: \homset{}{B}{D} \to \homset{}{A}{D}\;\text{such that }\lambda f D j = j\circ f\;\text{for } j\in \homset{}{B}{D}\\
				\lambda  g D&: \homset{}{C}{D} \to \homset{}{B}{D}\;\text{such that }\lambda g D i = i\circ g\;\text{for } i\in \homset{}{C}{D}
	\end{split}
\end{equation*}
so that we get 
\begin{equation*}
	\begin{split}
		\lambda f D\circ \lambda gD i&=\lambda fD (\lambda gD(i))\\
		&= \lambda fD (i\circ g)\\
		&= i\circ g\circ f 
	\end{split}
\end{equation*}
and
\begin{equation*}
	\begin{split}
		\lambda g\circ f D i &= i\circ g\circ f.
	\end{split}
\end{equation*}
So
\[\lambda f \circ \lambda g = \lambda g\circ f\]
which is what we expect from a contravariant functor. Therefore, the Yoneda Embedding $ \lambda $ which \emph{embeds} each arrow of category $ \cat{C} $ to the natural transformation (arrow) in the functor category $ \Func{\cat{C}}{\cat{Set}} $ is a functor\footnote{The Proposition \ref{P-3} additionally proves that the Yoneda Embedding is Full and Faithful.}!
\end{proof}
\subsection{Products of Objects}
\[\text{\emph{	In a group regarded as a category with one object, the products don't exists unless the group has only one element.}}\]
\begin{proof}
	This can be seen by the following limit of a discrete diagram in the given group as a category, assuming the product exists in the group with more than one element, so that the following commutes:
	\[\begin{tikzcd}
		& {*\times * = *} \\
		{*} && {*} \\
		& {*}
		\arrow["{f_k = p_1}"', from=1-2, to=2-1]
		\arrow["{f_l=p_2}", from=1-2, to=2-3]
		\arrow["{f_g\times f_h}"{description}, dashed, from=3-2, to=1-2]
		\arrow["{f_g}", from=3-2, to=2-1]
		\arrow["{f_h}"', from=3-2, to=2-3]
	\end{tikzcd}\]
	where $ *\times * = * $ because there is only one object in the whole category and we regard $ f_k $ and $ f_l $ as the projection arrows of the product object. Since the above diagram commutes, therefore,
	\begin{equation*}
		\begin{split}
			f_k \circ (f_g\times f_h) &= f_g\\
			f_l \circ (f_g \times f_h) &= f_h
		\end{split}
	\end{equation*}
	where $ f_g $ and $ f_h $ are any arrows. Suppose $ f_g = f_h = \Id{*}$. Hence, 
	\begin{equation*}
		\begin{split}
			f_g\times f_h &= f_{\inv{k}}\\
			f_g\times f_h &= f_{\inv{l}}.
		\end{split}
	\end{equation*}
	Therefore,
	\[ f_{\inv{k}} = f_{\inv{l}} \]
	which implies that
	\[f_k = f_l.\]
	But we initially fixed our choice of $ f_k $ and $ f_l $ (note they are components of the limit of the discrete diagram), therefore if we take any arbitrary $ f_g $ and $ f_h $ such that $ f_g\neq f_h $, since we know above diagram would commute for this, we get that $ f_g = g_h $ since $ f_g\times f_h $ is unique. Hence, we have a contradiction to the assumption that group has more than one element and product still exists. Hence product exists only in group with one element.
\end{proof}
\[\text{\emph{The Direct Product of two groups is the product in $ \cat{Group} $.}}\]
\begin{proof}
	
\end{proof}
\subsection{Internal Equivalence Relation in $ \cat{Set} $.}
\label{A-7}
\emph{In Diary, $ 14^{th} $ April, 2020.}
\newpage
\subsection{The Freyd's Adjoint Functor Theorem}
\label{A-8}
This is really an interesting proof with a lot of working parts! I proved it by following the exercise titled (GAFT) of the textbook. I am writing here only the intermediary results of the exercise and show only the final conclusion and an interesting intermediary result in detail here.
\begin{enumerate}
	\item { Suppose $ U :\cat{D} \longrightarrow \cat{C} $ is a functor. Then,\[U:\cat{D}\longrightarrow\cat{C}\text{ has an Adjoint} \;\bm{\iff}\;\forall\;C\in \obj{\cat{C}} \text{\;,  }(C,U) \text{ has an Initial Object.}\]
		\begin{proof}
			\emph{In Diary at $ 25^{th} - 22^{nd}$ February, $ 2020 $.}
	\end{proof}}
\item {Suppose $ U :\cat{D} \longrightarrow \cat{C} $ is a functor. Then,
\[\text{$ \cat{D} $ has all Limits and $ U $ Preserves Limits }\bm{\implies} \text{ $ (C,U) $ has all Limits.}\]
\begin{proof}
	\emph{In Diary at $ 21^{st} - 18^{th}$ February, $ 2020 $.}
\end{proof}
}
\item {Suppose $ U : \cat{D} \longrightarrow\cat{C} $ is a functor. Then,
\[\text{Functor $ U $ satisfies Solution Set Condition }\bm{\iff}\;\forall \;C\in \obj{\cat{C}}\text{ ,  $ (C,U) $ has a Weak Initial Set. }\footnote{A weak initial set is a small set of objects such that for any object of the underlying category, there exists an arrow with that object as the target and some object in the weak initial set as source.}\]
\begin{proof}
	\emph{In Diary at $ 17^{th} - 14^{th}$ February, $ 2020 $.}
\end{proof}}
\item {Suppose $ \cat{C} $ is a category. Then,
\[\text{$ \cat{C} $ has all Products and a Weak Initial Set }\bm{\implies} \text{ $ \cat{C} $ has a Weak Initial Object.}\]
\begin{proof}
	\emph{Trivial, but still in Diary at $ 13^{th}$ February, $ 2020 $.}
\end{proof}
}
\item {Suppose $ \cat{C} $ is a category. Then,
\begin{align*}
	\textbf{1. } & \cat{C}  \text{ has a Weak Initial Object, denoted $ A $.}\\
	\textbf{2. } & \text{ The Equalizer $ E $ of all endomorphisms of $ A $ exists, denoted $ f : E\to A $.}\\
	&\\
	\bm{\implies} & \text{$ E $ is the Initial Object of $ \cat{C} $.}
\end{align*}
\begin{proof}
	This is a slightly non-trivial result, hence I am formally writing it here. \\
	Take any object $ C\in \obj{\cat{C}} $. Since $ A $ is weakly initial, therefore $ \exists \;g: A\to C $ which may not be unique. Now, since $ f : E\to A $ is an Equalizer of all endomorphisms of $ A $, then, for any $ i_1, i_2 : A\to A $, we must have:
	\[i_1 \circ f = i_2 \circ f = \Id{A} \circ f = f.\]
	Now, to prove that arrows from $ E $ to any other object is unique, we take two arrows $ x,y : E\to C $. Let's take Equalizer of the pair $ x,y : E\to C $, and denote it by $ e : F\to E $. We then have the following diagram:
	\[\begin{tikzcd}
		F \\
		E & C \\
		A & A
		\arrow["x", shift left=2, from=2-1, to=2-2]
		\arrow["y"', shift right=2, from=2-1, to=2-2]
		\arrow["e"', tail, from=1-1, to=2-1]
		\arrow["f"', tail, from=2-1, to=3-1]
		\arrow["h", curve={height=-18pt}, from=3-1, to=1-1]
		\arrow[""{name=0, anchor=center, inner sep=0}, curve={height=12pt}, from=3-1, to=3-2]
		\arrow[""{name=1, anchor=center, inner sep=0}, curve={height=-12pt}, from=3-1, to=3-2]
		\arrow[shorten <=3pt, shorten >=3pt, Rightarrow, dotted, no head, from=1, to=0]
	\end{tikzcd}\]
where $ e $ and $ f $ are monomorphisms because they are equalizers (Proposition \ref{P-8}) and $ h $ is a non-unique arrow due to weak intiality of $ A $. 
Now we see that $ e\circ h\circ f : E\to E $ and $ f\circ e\circ h : A\to A $ are two arrows. Now, we should have:
\begin{align*}
	(f\circ e\circ h) \circ f &= f\\
	f\circ (e\circ h\circ f)&=f\circ \Id{E}\\
	\implies\;\;\;\;\;e\circ h \circ f &= \Id{E}&&\text{$ \because f $ is a mono} 
\end{align*}
Hence,
\begin{align*}
	x&=x\circ(e\circ h\circ f)\\ &= (x\circ e)\circ h\circ f\\
	&= (y\circ e)\circ h\circ f&&\text{$ \because e$ is the Equalizer.}\\
	&= y\circ (e\circ h\circ f)\\
	&= y\circ \Id{E} = y 
\end{align*}
That is, $ x = y $. Now since $ C\in \obj{\cat{C}} $ was arbitrary, therefore $ E $ is initial.
\end{proof}

}
\item {\textbf{General Adjoint Functor Theorem - Conclusion}.\begin{proof}
		 (L $ \implies $ R) Suppose $ R: \cat{D} \longrightarrow\cat{C} $ has a Left Adjoint and $ \cat{D} $ has all Limits. By \textbf{Part (1)}, $ \forall C\in \obj{\cat{C}} $, the comma category $ (C,R) $ has an Initial Object. By \textbf{Part (3)}, since we have that $ (C,R) $ has a weak initial set as singleton of it's initial object, then $ R:\cat{D}\longrightarrow\cat{C} $ satisfies solution set condition. The fact that $ R : \cat{D}\longrightarrow\cat{C} $ preserves limit is obvious from the fact $ R $ has an Adjoint (Theorem \ref{T-5}).\\\\
		(R$ \implies $ L) Suppose $ R:\cat{D} \longrightarrow\cat{C}$ preserves all Limits and $ R $ satisfies solution set condition. Now since $ R $ satisfies solution set condition, therefore by \textbf{Part (3)}, $ \forall C\in \obj{\cat{C}} $, the comma category $ (C,R) $ has a weak initial set. Now because $ R $ preserves limits and $ \cat{D} $ has all Limits, by \textbf{Part (2)}, $ \forall C\in \obj{\cat{C}} $, $ (C,R) $ is complete and hence has all products. Now because $ (C,R) $ has all products and $ (C,R) $ has a weak initial set, hence by \textbf{Part (4)}, $ (C,R) $ has a weak initial object.\\
		Finally, by \textbf{Part (1)}, since $ (C,R) $ has an initial object $ \forall C\in \obj{\cat{C}} $, we conclude that $ R : \cat{D} \longrightarrow\cat{C}$ has a Left Adjoint.
	\end{proof}
}
\end{enumerate}
\[\]
%\subsection{An Interesting Functor (Sheaves!).}
%\label{A-6}
%
%
%\subsection{Godement's Rules}
%\label{A-7}
%Given : We have the following situation:
%\[\begin{tikzcd}
%	{\mathfrak{B}} && {\mathfrak{C}} && {\mathfrak{D}}
%	\arrow[""{name=0, anchor=center, inner sep=0}, "F",  shift left=2, curve={height=-12pt}, from=1-1, to=1-3]
%	\arrow[""{name=1, anchor=center, inner sep=0}, "G"', shift right=2, curve={height=12pt}, from=1-1, to=1-3]
%	\arrow[""{name=2, anchor=center, inner sep=0}, "H", shift left=2, curve={height=-12pt}, from=1-3, to=1-5]
%	\arrow[""{name=3, anchor=center, inner sep=0}, "K"', shift right=2, curve={height=12pt}, from=1-3, to=1-5]
%	\arrow["\kappa", shorten <=4pt, shorten >=4pt, Rightarrow, from=0, to=1]
%	\arrow["\mu", shorten <=4pt, shorten >=4pt, Rightarrow, from=2, to=3]
%\end{tikzcd}\]
%\emph{a.} To show 
\newpage
\epigraph{
	To step from the symbol to that which is symbolized, though this does afford a peculiarly exacting demand upon acuity of thought, yet requires much more. Here, feeling, in the best sense, must fuse with thought. The thinker must learn also to feel his thought, so that, in the highest degree, he thinks devotedly. It is not enough to think clearly, if the thinker stands aloof, not giving himself with his thought. The thinker arrives by surrendering himself to Truth, claiming for himself no rights save those that Truth herself bestows upon him. In the final state of perfection, he possesses no longer opinions of his own nor any private preference. Then Truth possesses him, not he, Truth. He who would become one with the Eternal must first learn to be humble. He must offer, upon the sacrificial alter, the pride of the knower. He must become one who lays no possessive claim to knowledge or wisdom. This is the state of the mystic ignorance -- of the emptied heart. }{Franklin Merrell-Wolff, Pathways Through To Space, 1973.}
\begin{equation*}
	\begin{split}
		\text{Trivial Things must be Trivially Trivial.}&\\
		&\text{-Tenet of Category Theory}
	\end{split}
\end{equation*}
\end{document}

